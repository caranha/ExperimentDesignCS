
\section{Course Review}
\subsection{Outline}
\begin{frame}{Lecture Outline}
  Let's review the process of experiment design that was studied in this course

  \begin{itemize}
    % \item Motivating Example
    \item Hints for Designing your Experiment
    \item Hints for Statistical Analysis
  \end{itemize}
\end{frame}

% TODO: Create a better motivating example
\begin{frame}{Motivating Example}
  In my research, I need to take pictures of plants, and then process them using a segmentation network.\bigskip

  The camera creates very high resolution pictures in RAW format. However, my network reads the picture in a lower resolution format.\bigskip

  I have several choices for the transformation of the picture:
  \begin{itemize}
    \item What algorithm I should use to scale down the picture (linear scaling, cubic scaling, etc);
    \item What final resolution to save the pictures as.
  \end{itemize}\bigskip

  I have to choose the scaling algorithm and the final resolution. These choices will effect the time to convert the pictures, and the amount of disk space needed.
  The choice may also have an effect on the performance of the algorithm.\bigskip

  What should I do?
\end{frame}


% \begin{frame}{Example: Image Conversion}
%   % # How to design an experiment
%   %
%   % ############################################################
%   % # Example
%   % Comparison of conversion of image from Raw to PNG/TIFF
%   %
%   % The idea of the comparison between image formats is interesting.  Please think about the following questions:
%   %
%   % - Do you care about the result of this experiment? Why?
%   % - Does the content of the image affect the result of the experiment? How? How can you control the effect of the contents of the pictures? What kind of pictures are you interested in?
%   % - Why did you choose 3 pictures?
%   % - Why did you choose the 5% difference?
% \end{frame}

\subsection{Hints for Designing the Experiment}

\begin{frame}{Point 1: Choose an experiment that you care about}

  If you don't care about the question/answer of your experiment, it will be very hard to do a good experiment.\bigskip

  \begin{itemize}
    \item You won't be able to imagine what results are important or irrelevant;
    \item You will not be able to think about the factors that can influence the experiment.
    \item You will not be able to decide which factors to fix or vary; and the parameters of variation.
  \end{itemize}
\end{frame}

\begin{frame}{Point 2: Think about the output variable}
  The Output variable is the result of your experiment. It is the value that you are interested in observing.\bigskip

  When you think about the output variable, consider the characteristics of this variable.\bigskip

  \begin{itemize}
    \item Is the variable discrete or continuous?
    \item What is the range of values for this variable? What are the typical values?
    \item What is the typical variation for this variable?
    \item What is the size of difference that is important for you and for your experimental question?
  \end{itemize}\bigskip

  The answer to some of the questions above may come from your experience. The answer to some of the questions above can be "I don't know".
  That is okay, we make experiments and analyse data to find the answers to these questions.\bigskip

  If you have multiple output variables, you might also want to think if you want to make separate experiments.
\end{frame}

\begin{frame}{Point 3: The experimental Factors}
  The experimental factors are the conditions that could change the result of the experiment. \bigskip

  When you think about each experimental factor, you have to consider their characteristics as well.\bigskip

  \begin{itemize}
    \item Are these factors discrete or continuous?
    \item What is the range of possible values? What are the typical values?
    \item What is the relationship between the factor and the output variable?
    \item What are the values of the factor that interest you?
  \end{itemize}\bigskip
\end{frame}

\begin{frame}{Point 3: The experimental Factors (2)}

  Based on these characteristics, you should separate the factors as follows:\bigskip

  \begin{itemize}
    \item Factors that you will study in the experiment. These factors you will set to specific values, because you are interested in the influence of these factors in the outcome of the output variable.
    \item Factors that you want to control in the experiment. These factors have important influence in the output variable, but you do not want to study their influence. So you will have to fix their value during the experimental design, or limit their effect using techniques such as pairing (studied in class) or blocking (not studied in class)
    \item Other factors, that you cannot control, but you can estimate their influence in the output variable.
  \end{itemize}

\end{frame}

\begin{frame}{Point 4: Expected results}
  It is helpful to try to predict what is the expected result of the experiment before you execute the experiment. When you learn the result, you can more clearly define if the result confirmed your expectations, or revealed something surprising.\bigskip

  By considering the expected result, you can discover if you have missing knowledge about your experiment that you need to study, and if there are parts of the experiment design that are missing.
\end{frame}

\subsection{Hints for Data Collection}

\begin{frame}{Point 5: Pre-experimental choices}
  Before you begin your experiment, you should make several choices:\bigskip

  \begin{itemize}
    \item Decide the Statistical Test (Depends on type of output variable, number of factors and factor levels, characteristics of the problem);
    \begin{itemize}
      \item One sample: z-test or t-test;
      \item two samples: t-test, paired or not paired;
      \item multiple samples: ANOVA;
      \item output variable is not a normal random variable: non-parametric tests;
    \end{itemize}
    \item Define Hypothesis and critical values;
    \item Decide the Test Parameters (confidence, power, minimal difference of interest), depend on the experiment characteristics;
    \item Calculate sample size;
  \end{itemize}
\end{frame}

\begin{frame}{Point 6: Collecting Data and Descriptive Statistics}
  During data collection, make sure to pay attention to any unusual things that happen during the experiment.\bigskip

  Describe your initial results using descriptive statistics, such as mean and standard deviation of the output variable, boxplots, confidence interval, etc.\bigskip

  A good visualization here can make the next step much easier.
\end{frame}

\begin{frame}{Point 7: Statistical test analysis and conclusions}
  Perform your statistical test, calculate the test statistic and the p-value.\bigskip

  Make any necessary calculations to verify that your output variable follows the necessary assumptions of the statistical test.\bigskip

  When reporting the result, do not forget to include not only the p-value, but also the descriptive statistics.\bigskip

  Compare the final result with your initial prediction. What did you learn?
\end{frame}

% TODO: Complete this
% ## Point 7: Statistical test analysis and conclusions
%
% ## Point Extra: Ethical Research
% - Iris dataset:
%   - Very used in ML
%   - Not really an interesting dataset
%   - Originally published in the "Annals of Eugenics"
%   - Consider the origins of your dataset.
%   - More complex issues, such as equality, etc.
