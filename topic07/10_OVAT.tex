\section{One Variable at a Time}
\begin{frame}
  \begin{center}
    {\bf Part I -- One Variable At a Time (OVAT)}
  \end{center}
\end{frame}

\begin{frame}{Factors and Levels}
  \begin{itemize}
    \item {\bf Factors}: Input variables in the experiment / "control variables";
    \item {\bf Levels}: The values that a factor can assume / "treatments";
  \end{itemize}

  \begin{block}{Example: Carrying Speed Experiment}
    I want to measure how the weight of my luggage affects my speed. I measure
    how far I walk in 5 minutes, when carrying 10, 20 and 40kgs of luggage.\bigskip

    \begin{itemize}
      \item Response Variable: distance walked in 5 minutes;
      \item Factors: Weight of luggage;
      \item Levels: 10kgs, 20kgs, 40kgs;
    \end{itemize}
  \end{block}
\end{frame}


\begin{frame}{Experiments with One Factor}
  Until now, we considered experiments with 1 factor and 1 return variable;
  \begin{itemize}
    \item {\bf t-test}: 1 factor, two levels
    \item {\bf ANOVA}: 1 factor, many levels
  \end{itemize}

  \begin{block}{Example}
    We want to compare how long it takes to train a neural network until a certain
    error value is reached. We compare four optimization algorithms: SGD, Adam, Adagrad, CMA-ES.

    \begin{itemize}
      \item Response Variable: Time until error threshold is reached;
      \item Factor: Optimization Algorithm;
      \item Levels: 1- SGD, 2- Adam, 3- Adagrad, 4- CMA-ES
    \end{itemize}
  \end{block}
\end{frame}

\begin{frame}{Experiment with Two Factors}
  \begin{block}{
    We add a new factor to the experiment: The number of layers
    in the network (2, 5 or 10)}
    \begin{itemize}
      \item Response Variable: Time until error threshold is reached;
      \item Factor 1, 4 levels: Optimization Algorithm: SGD, Adam, Adagrad, CMA-ES;
      \item Factor 2, 3 levels: 2, 5, 10 layers;
    \end{itemize}
  \end{block}

  \begin{itemize}
    \item {\bf Experiment with 1 factor:} Only consider the effect of the levels on the response variable.
    \item {\bf Experiment with 2 factors:} Consider main effect and interference effects:
    \begin{itemize}
      \item {\bf Main Effect:} Effect of the levels of each factor in the response variable.
      \item {\bf Interaction Effects:} Combined effects of levels of both factors in the response variable.
    \end{itemize}
  \end{itemize}

\end{frame}

\begin{frame}{Number of Factors and Experiment Complexity}

  The higher the number of factors and levels, the more complex becomes the experiment.
  \begin{itemize}
    \item 2 factors, 4 levels each: 16 total combinations; 2nd order interactions;
    \item 3 factors, 3 levels each: 27 total combinations; 3rd order interactions;
    \item 4 factors, 3 levels each: 81 total combinations; 4th order interactions;
    \item ...
  \end{itemize}\bigskip

  Because of this, it is desirable to keep experiments simple, even if there are statistical techniques to detect interaction effects.
\end{frame}

\begin{frame}{OVAT -- One Variable at a Time}
  The simplest design to use with multiple factors is the OVAT Design.\bigskip

  \begin{itemize}
    \item For $N$ factors, perform $N$ different experiments, one for each factor.
    \item The levels of the other factors are fixed at a "standard" value.
    \item For each factor, you perform the statistical analysis independently (t-test, anova, visualizations, etc);
  \end{itemize}
\end{frame}

\begin{frame}{OVAT -- Example}
  \begin{block}{Fine tuning Cr and F for Differential Evolution (DE)}
    DE is an effective optimization algorithm where the performance depends
    strongly on two parameters: Cr and F.\bigskip

    To find the optimal values for these parameters, you perform a preliminary
    experiment on a synthetic benchmark.
  \end{block}

  OVAT Design:
  \begin{itemize}
    \item Choose standard values for Cr and F from the literature.
    \item Analyze Cr using an experiment. Fix the value of F and try 10
      different values (levels) for CR.
    \item Analyze F using another experiment. Fix the value of Cr and try 10
      different values (levels) for F.
    \item The execution and analysis of these two experiments follow the
      methods that we already studied.
  \end{itemize}
\end{frame}




























%
