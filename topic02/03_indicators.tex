
\section{Basic Concepts}
\begin{frame}
  \begin{center}
    Part II.a: Indicators Basic Concepts
  \end{center}
\end{frame}

\begin{frame}{Lecture Outline}

  In the first lecture, we talked about what is science, and how experiments
  (carefully designed experiments) are important in science.\bigskip

  Starting from this lecture, we will talk about how we use statistics to
  understand the data that we gather from experiments, and how we can draw
  conclusions about them.\bigskip

  Topics for today:
  \begin{itemize}
    \item \structure{Experimental Data}: Population, Observation and Sample;
    \item \structure{Point indicators}: How we obtain information about the population from samples;
    \item \structure{Interval indicators}: Indicators with quality info!
  \end{itemize}
\end{frame}

\begin{frame}{Lecture Outline}{Data and "Experiment Data"}
  When we talk about "data" in Computer Science, the first thing that comes to mind is "information that we feed to a program". {\bf For example}: images, network logs, user databases, etc.\bigskip

  In this course, we are talking about "Experiment Data", which should be understood as "Data about the result of an experiment". {\bf For example}: How long did the experiment take? What is the success rate of my program?\bigskip

  In fact, we can use the tecniques of this lecture for the first kind of data too! But to make things simpler, let's concentrate on the second kind of data.
\end{frame}

\begin{frame}{Example of Data Collection}{Using Experiment Data to characterize a system}
  \begin{columns}
    \column{0.8\textwidth}
    How can we use experiment data to learn more about the world?\bigskip

    \structure{Tsukuba University} is famous for many olympic athletes. Let's say
    I want to investigate {\bf WHY}. After thinking a bit, I come up with two questions:
    \begin{itemize}
      \item Is the olympic performance related with the body of the students?
      \item Are students in Tsukuba stronger / taller / healthier than other unis?
    \end{itemize}\bigskip

    The 1st question seems hard to answer, but maybe the 2nd is easier?\bigskip

    Imagine I take one student from Tsukuba and measure their height.
    Can this information help me answer the second question?
    \column{.2\textwidth}
    \includegraphics[width=\textwidth]{../img/irasutoya_height}
    \ppagenote{Height image from \url{https://www.irasutoya.com}}
  \end{columns}
\end{frame}

\begin{frame}{Using experiment data to characterize a system}
  From the height of only one student, I cannot really learn anything about
  the height of the students of the university {\bf in general}.
  A better approach would be to measure the height of several students.\bigskip

  \begin{center}
    \includegraphics[width=0.09\textwidth]{../img/irasutoya_height}
    \includegraphics[width=0.1\textwidth]{../img/irasutoya_height}
    \includegraphics[width=0.08\textwidth]{../img/irasutoya_height}
    \includegraphics[width=0.09\textwidth]{../img/irasutoya_height}
    \includegraphics[width=0.1\textwidth]{../img/irasutoya_height}
    \includegraphics[width=0.1\textwidth]{../img/irasutoya_height}
    \includegraphics[width=0.09\textwidth]{../img/irasutoya_height}
    \includegraphics[width=0.1\textwidth]{../img/irasutoya_height}
  \end{center}
  \bigskip

  Now that I have the height of many students, what can I say about the height
  of the students in the university, {\bf in general}?
\end{frame}

\subsection{Basic Concepts}
\begin{frame}{Population and Sample}
  This example introduces us to some important concepts in statistics:
  \structure{Population, Observation, and Sample}.\bigskip

  \begin{itemize}
    \item {\bf Population}: This is a set of objects that we want to learn more about, using experiments. It can be a real set (all students of the university), or a theoretical set (all possible results of running a program).

    \item {\bf Observation}: This is one element from the population. One student from the university, or one execution of the program. One data point from an experiment.\medskip

    \item {\bf Sample}: This is a set of observations. A subset of the population.
  \end{itemize}
  \bigskip

  Our goal, when we analyse data from an experiment, is to {\bf "learn something about the population, by examining the observations in the sample"}.
\end{frame}

\begin{frame}{Population and Sample}{Learn something about the population, by examining the observations in the sample}
  \begin{block}{Population}
    \begin{columns}
      \column{.18\textwidth}
      \includegraphics[width=1\textwidth]{../img/ballpool}
      \column{.65\textwidth}
      You want to know the proportion of colorful balls in a pool (you like the red ones). Because we don't know exactly how many there are, we need to {\bf make an estimation}.
    \end{columns}
  \end{block}
  \begin{block}{Sample}
    \begin{columns}
      \column{.14\textwidth}
      \includegraphics[width=1\textwidth]{../img/ballhand}
      \column{.65\textwidth}
      To learn the proportion of red balls, we pick a number of balls, and {\bf estimate that the proportion of red balls in the pool is equal to the proportion in my hand}.
    \end{columns}
  \end{block}
\end{frame}

\begin{frame}{Population, Model and Parameters}{What is a model?}
  A model is a description of the population, focused on the scientific questions that we want to make.\bigskip

  \begin{itemize}
    \item The balls are distributed evenly in the pool, so I can take my sample from any part.
    \item The height of the students in the university can be represented by a {\bf normal curve}, with mean $\mu$ and standard deviation $sd$.
    \item The SIR infection model says that {\bf susceptible} people become {\bf infected}, and then {\bf Recovering}, so if we can learn the \structure{number of people in each group} and the \structure{transition probability}, we can predict the progress of a disease.
  \end{itemize}\bigskip

  The goal of many experiments is to use data to {\bf estimate the parameters of a model}.
\end{frame}

\begin{frame}{Population, Model and Parameters}{Example of model parameters}
  \begin{columns}
    \column{0.5\textwidth}
      The usual goal of the analysis of experiment data is to estimate the
      values of the parameters of the model, from data in the sample.\bigskip

      The true value ({\bf unknown}) of the parameters in the model is usually called $\theta$. The value that we estimate from the experiment, is usually called $\hat{\theta}$.\bigskip

      The {\bf model} must be determined during the experiment design phase.
      A bad model may lead to wrong conclusions from the data...
    \column{0.5\textwidth}
    \includegraphics[width=.8\textwidth]{../img/wikipedia_triceratops}
    \ppagenote{Triceratops information table CC by Zachi Evenor and MathKnight}
  \end{columns}
\end{frame}



\begin{frame}{Samples and Statistics}
  By observing data obtained from a sample, we can {\bf characterize} (estimate) parameters from a population of interest. For example:\bigskip

  \begin{itemize}
    \item We calculate the average of the running time of multiple executions of a program, and estimate the mean running time;
    \item We ask the age of several students in a school, and estimate the maximum and minimum age of the students;
    \item We estimate the efficacy of a remedy by counting what percentage of patients get better after drinking it;
    \item We determine which of two neural networks is more precise by subtracting the test error of the two networks from each other;
  \end{itemize}\bigskip

  \begin{block}{Statistic}
    A statistic is a {\bf function} calculated from data obtained from a sample.
  \end{block}
\end{frame}

\begin{frame}{Point and Interval Indicators}
  The idea of estimating parameters of a population using information
  obtained from a sample is called {\bf statistical inference}.
  \bigskip

  In this lecture, we will focus on two central concepts of statistical inference: {\bf Point Estimators} and {\bf Interval Estimators}.\bigskip

  \begin{itemize}
    \item {\bf Point Estimators}: are statistics that estimate the value of a population parameter from information in a sample;

    \item {\bf Interval Estimators}: are statistics that estimate a \structure{range of values} of a population parameter from information in a sample;
  \end{itemize}
\end{frame}

\begin{frame}{Statistics and Sampling Distributions}
  Suppose that you want to obtain a point estimate for an arbitrary parameter of the population (e.g. mean size);\medskip

  Random samples of the population can be interpreted as a {\bf random variable}, and any function of these samples (any statistic) will
  be a random variable as well.\medskip

  \begin{columns}
    \column{.3\textwidth}
      \includegraphics[width=1\textwidth]{../img/sampling_distribution}
    \column{.7\textwidth}
      As a random variable, any statistic also has its own \structure{probability distribution}, called {\bf sampling distribution}.\bigskip

      Most statistical tests use properties of the sampling distributions (which are not the same as the true distribution of the population). We will talk more about those later.
  \end{columns}

\end{frame}

%% TODO: this part was taken directly from campelo's materials, Needs update.
\section{Point Estimators}
\section{Basic Concepts}
\begin{frame}
  \begin{center}
    Indicators, Part II - Point Estimators
  \end{center}
\end{frame}

\subsection{Definition}
\begin{frame}{Definition of Point Indicator}
  A \emph{point estimator} is a statistic which provides the value of maximum plausibility for an (unknown) population parameter $\theta$.
  \bigskip

  Consider a random variable $X$ distributed according to a given $f(X|\theta)$.\bigskip

  Consider also a random sample from this variable: $x=\{x_1,x_2,\ldots,x_n\}$;\bigskip

  A given function $\hat{\Theta}=h\left(x\right)$ is called a \emph{point estimator} of the parameter $\theta$, and a value returned by this function for a given sample is referred to as a \emph{point estimate} $\hat{\theta}$ of the parameter.
\end{frame}

%% TODO: this part was taken directly from campelo's materials, Needs update.
\begin{frame}{Examples}
Point estimation problems arise frequently in all areas of science and engineering, whenever there is a need for estimating, e.g.,:\bigskip

\begin{itemize}
  \item a population  mean, $\mu$;
	\item a population variance, $\sigma^2$;
	\item a population proportion, $p$;
	\item the difference in the means of two populations, $\mu_1-\mu_2$;
	\item etc..
\end{itemize}\bigskip

In each case there are multiple ways of performing the estimation task, and the decision about which estimators to use is based on the mathematical properties of each statistic.
\end{frame}

\begin{frame}{Errors and Biases}
  Note that we are being very careful to always use the word {\bf estimate} when we talk about statistics. Why is that?
  \bigskip

  In all the examples that we mentioned, if we are unlucky\footnote{or careless, or malicious}, we could obtain an estimate that is very different from the true value of the population:
  \begin{alertblock}{Bad statistics example}
    To estimate the height of the students of a school, we pick 10 students, and we measure the height of the youngest one.
  \end{alertblock}\medskip
  \begin{itemize}
    \item {\bf Error}: The difference between an estimate and the true value of a population's parameter;
    \item {\bf Bias}: The property of a statistic that systematically produces wrong estimates;
  \end{itemize}
\end{frame}

%% TODO: this part was taken directly from campelo's materials, Needs update.
\begin{frame}{Unbiased estimators}
A good estimator should consistently generate estimates that lie close to the real value of the parameter $\theta$.\bigskip

A given estimator $\hat{\Theta}$ is said to be \textit{unbiased} for parameter $\theta$ if:\bigskip

\begin{equation*}
E\left[\hat{\Theta}\right] = \theta
\end{equation*}
\noindent or, equivalently:
\begin{equation*}
E\left[\hat{\Theta}\right] - \theta = 0
\end{equation*}\bigskip

The difference $E\left[\hat{\Theta}\right] - \theta$ is referred to as the \textit{bias} of a given estimator.
\end{frame}

%% TODO: this part was taken directly from campelo's materials, Needs update.
\begin{frame}{Unbiased estimators}
For example, the usual estimators for mean is an unbiased estimator;
\bigskip

Let $x_1,\ldots,x_n$ be a random sample from a given population $X$, characterized by its mean $\mu$ and variance $\sigma^2$. In this situation, it is possible to show that:
\bigskip

\begin{equation*}
E\left[\bar{x}\right] = E\left[\frac{1}{n}\sum\limits_{i=1}^{n}x_i\right] = \mu
\end{equation*}
\vfill

(Remember that the expected value of one observation is the mean value of the population)

\end{frame}

%% TODO: this part was taken directly from campelo's materials, Needs update.
\begin{frame}{Unbiased estimators}

Fora a population parameter $\theta$, it is usually possible to define more than one unbiased estimator. The variances of these estimators may, however, be different\bigskip

\begin{columns}[T]
    \column{0.5\textwidth}\vspace{-1.5em} \includegraphics[width=1\textwidth]{../img/unbiased_variance.png}
    \column{0.5\textwidth} In these cases, we usually want to obtain the unbiased estimator of minimal variance. This is generally called the \textit{minimal-variance unbiased estimator} (MVUE).
\end{columns}\bigskip

MVUE are generally chosen as estimators due to their ability of generating estimates $\hat{\theta}$ that are (relatively) close to the real value of $\theta$.
\ppagenote{Image: D.C.Montgomery,G.C. Runger, \textit{Applied Statistics and Probability for Engineers},Wiley 2003.}
\end{frame}


%% TODO: this part was taken directly from campelo's materials, Needs update.
\begin{frame}{Standard error of a point estimator}
Remember that because a point estimator is a random variable, it has an associated distribution and error. For example, the standard error of an estimator $\hat{\Theta}$ is
\begin{equation*}
\sigma_{\hat{\Theta}} = \sqrt{Var\left[\hat{\Theta}\right]}
\end{equation*}\bigskip

However, we can't know this directly. We can {\bf estimate} the standard error of the estimator from the data in the sample. In this case, we refer to it as the \textit{estimated standard error}, $\hat{\sigma}_{\hat{\Theta}}$ (the notations $s_{\hat{\Theta}}$ and $se(\hat{\Theta})$ are also common).

\hfill\includegraphics[width=.1\textwidth]{../img/yodawg}
\end{frame}

%% TODO: this part was taken directly from campelo's materials, Needs update.
\begin{frame}{Standard error of a point estimator}{Examples}
  Assuming a random variable $X$ from a gaussian distribution, and a sample error $s$, we can calculate the standard errors of several common point indicators\footnote{See Ahn and Fessler (2003), \textit{Standard Errors of Mean, Variance, and Standard Deviation Estimators}: \url{https://git.io/v5Z5v}}

\begin{equation*}
\hat{\sigma}_{\bar{X}} = \frac{s}{\sqrt{n}}
\end{equation*}

\begin{equation*}
\hat{\sigma}_{S^2} = s^2\sqrt{\frac{2}{n-1}}
\end{equation*}

\begin{equation*}
\hat{\sigma}_{S} = \frac{s}{\sqrt{2(n-1)}} + O\left(\frac{1}{n\sqrt{n}}\right)\approx \frac{s}{\sqrt{2(n-1)}}
\end{equation*}\bigskip
\end{frame}

\begin{frame}{Point Estimator Use Case}
  \includegraphics[width=.4\textwidth]{../img/pixabay_cable}
  \ppagenote{Coaxial cable image from \url{https://pixabay.com}}
  \bigskip

  Consider an operation to produce coaxial cables\footnote{Example inspired on \url{https://www.sas.com/resources/whitepaper/wp_4430.pdf}}. The mean resistance of the production is $50\Omega$, with a standard deviation of $2\Omega$ (Population mean, and population deviation).\bigskip

  Let's assume that the resistance value of the produced cables are distributed follow a normal distribution ($X\sim\mathcal{N}\left(\mu=50,\sigma^2=4\right)$)
\end{frame}

\begin{frame}{Point estimator use case}
  \begin{columns}
    \column{.35\textwidth}
    \includegraphics[width=1\textwidth]{../img/pixabay_cable}
    \column{.65\textwidth}
    Suppose that we take a random sample of $25$ cables is taken from
    this production process (an experiment, to measure if the process
    is correct, for example).
  \end{columns}
  \bigskip
The \structure{sample mean} of the the observations is:
\begin{equation*}
\bar{x} = \frac{1}{25}\sum\limits_{i=1}^{25}{x_i}
\end{equation*}

The \structure{sample mean} follows a normal distribution, with $E[\bar{x}] = \mu = 50\Omega$\footnote{since the sample mean is an unbiased estimator} and $\sigma_{\bar{x}} = \sqrt{\sigma^2/25} = 0.4\Omega$. The error depends on the sample size.
\end{frame}


\subsection{The Central Limit Theorem}

%% TODO: this part was taken directly from campelo's materials, Needs update.
\begin{frame}{The Central Limit Theorem}
In the previous example, the production operation followed a normal distribution. But even for a population with an arbitrary distribution, the sampling distribution of its mean tends to be approximately normal.
%(with $E[\bar{x}] = \mu $ and $s_{\bar{x}} = \sigma^2/n$).
\bigskip

More generally, let $x_1,\ldots,x_n$ be a sequence of \textbf{independent and identically distributed} (\textbf{iid}) random variables, with mean $\mu$ and finite variance $\sigma^2$. Then:
\begin{equation*}
z_n = \frac{\sum\limits_{i=1}^{n}{(x_i)} - n\mu}{\sqrt{n\sigma^2}} = \frac{\bar{x} - \mu}{\sqrt{\sigma^2/n}}
\end{equation*}

is distributed asymptotically as a standard Normal variable, that is, $z_n\sim\mathcal{N}(0,1)$.
\end{frame}

%% TODO: this part was taken directly from campelo's materials, Needs update.
\begin{frame}
{The Central Limit Theorem}
This result is known as the \textit{Central Limit Theorem}\footnote{For more details on the CLT, see \url{https://www.encyclopediaofmath.org/index.php/Central_limit_theorem}}, and is one of the most useful properties for statistical inference. The CLT allows the use of techniques based on the Normal distribution, even when the population under study is not normal.\bigskip

For ``well-behaved'' distributions (continuous, symmetrical, unimodal - the usual bell-shaped pdf we all know and love) even small sample sizes are commonly enough to justify invoking the CLT and using parametric techniques.\bigskip
\end{frame}

\begin{frame}
{Sampling Distributions}
{The Central Limit Theorem}
For an interactive demonstration of the CLT, download the files in {\small\url{https://git.io/vnPj8}} and run on RStudio.
\bigskip

{\centering\includegraphics[width=\textwidth]{../img/CLTdemo.png}}
\end{frame}

\section{Interval Estimators}
\section{Basic Concepts}
\begin{frame}
  \begin{center}
    Indicators, Part III - Interval Estimators
  \end{center}
\end{frame}

%% TODO: need to review the entire section

\begin{frame}{Statistical Intervals}
Statistical intervals are important in quantifying the uncertainty associated to a given estimate;
\bigskip

As an example, let's recap the coaxial cables example: \textit{a coaxial cable manufacturing operation produces cables with a target resistance of $50\Omega$ and a standard deviation of $2\Omega$. Assume that the resistance values can be well modeled by a normal distribution}.
\bigskip

Let us now suppose that a sample mean of $n=25$ observations of resistance  yields $\bar{x} = 48$. Given the sampling variability, it is very likely that this value is not exactly the true value of $\mu$, but we are so far unable quantify how much uncertainty there is in this estimate.
\end{frame}

%=====

\begin{frame}{Definition}
\textit{Statistical intervals} define regions that are likely to contain the true value of an estimated parameter.
\bigskip

More formally, it is generally possible to quantify the level of uncertainty associated with the estimation, thereby allowing the derivation of sound conclusions at predefined levels of certainty.
\bigskip

Three of the most common types of interval are:

\begin{itemize}
  \item Confidence Intervals;
  \item Tolerance Intervals;
  \item Prediction Intervals;
\end{itemize}
\end{frame}


%=====

\begin{frame}{How to Interpret a Confidence Interval}
Confidence intervals quantify the degree of uncertainty associated with the estimation of population parameters such as the mean or the variance.\bigskip

Can be defined as ``\textit{the interval that contains the true value of a given population parameter with a confidence level of $100(1-\alpha)$}'';\bigskip

Another useful definition is to think about confidence intervals in terms of confidence \textit{in the method}: ``The method used to derive the interval has a hit rate of $95\%$'' - i.e., the interval generated has a $95\%$ chance of `capturing' the true population parameter.''
\end{frame}

%=====

\begin{frame}
{Confidence Intervals}
{Example: 100 $CI_{.95}$ for a sample of 25 observations}
\centering\includegraphics[width=.8\textwidth]{../img/CIs.pdf}
\footnote{For an interactive demonstration of the factors involved in the definition of a confidence interval, Run the files at \url{https://git.io/vxXGj} on RStudio.}
\end{frame}

%=====
% WARNING: The lecture picks up too fast in the next three slides
% Break them up with examples and more text.
%=====


%% TODO -- TODO: z and t came out of nowhere here!!!
% Topics that the students don't know:
% Z distribution
% T distribution
% Degrees of freedom.
% Need to explain this (maybe when explaining the distribution
% of an indicator)
\begin{frame}{CI on the Mean of a Normal Variable}
The two-sided $CI_{(1-\alpha)}$ for the mean of a normal population with known variance $\sigma^2$ is given by:
\begin{equation*}
\bar{x}+z_{\alpha/2}\frac{\sigma}{\sqrt{n}}\leq\mu\leq\bar{x}+z_{1-\alpha/2}\frac{\sigma}{\sqrt{n}}
\end{equation*}
\noindent where $(1-\alpha)$ is the confidence level and $z_{x}$ is the $x$-quantile of the standard normal distribution.
\bigskip

For the more usual case with an unknown variance,
\begin{equation*}
\bar{x}+t_{\alpha/2}^{(n-1)}\frac{s}{\sqrt{n}}\leq\mu\leq\bar{x}+t_{1-\alpha/2}^{(n-1)}\frac{s}{\sqrt{n}}
\end{equation*}
\noindent where $t_{x}^{(n-1)}$ is the $x$-quantile of the t distribution with $n-1$ degrees of freedom.
\end{frame}

%=====

\begin{frame}{CI on the Variance and Standard Deviation of a Normal Variable}
A two-sided confidence interval on the variance of a normal variable can be easily calculated:
\begin{equation*}
\frac{(n-1)s^2}{{\chi^2}_{1 - \alpha/2}^{(n-1)}}\leq\sigma^2\leq\frac{(n-1)s^2}{{\chi^2}_{\alpha/2}^{(n-1)}}
\end{equation*}
\noindent where ${\chi^2}_{x}^{(n-1)}$ represents the x-quantile of the $\chi^2$ distribution with $n-1$ degrees of freedom. For the standard deviation one simply needs to take the squared root of the confidence limits.
\end{frame}

%=====

%\begin{ftst}
%	{Bootstrap Confidence Intervals}
%	{Using resampling}
%	Confidence intervals can also be constructed using a resampling technique called \textit{bootstrap};
%	\vone
%	This method works by resampling (with replacement) from the
%
%\end{ftst}

%\begin{ftst}
%{Prediction Intervals}
%{Definition}
%Prediction intervals quantify the uncertainty associated with forecasting the value of a future observation;
%\vone
%Essentially, one is interested in obtaining an interval within which he or she can declare that the next observation will fall with a given probability;
%\vone
%For a normal distribution, the tolerance interval for a single next observation (given an existing sample of size $n$) is:
%\beqs
%\bar{x}+t_{\alpha/2}^{(n-1)}s\sqrt{1 + \frac{1}{n}}\leq x_{n+1}\leq\bar{x}+t_{1-\alpha/2}^{(n-1)}s\sqrt{1 + \frac{1}{n}}
%\eqs
%\end{ftst}
%
%%=====
%
%\begin{ftst}
%{Tolerance Intervals}
%{Definition}
%``\textit{A tolerance interval is an \textbf{enclosure} interval for a specified proportion of the sampled population, not its mean or standard deviation. For a specified confidence level, you may want to determine lower and upper bounds such that a given percent of
%the population is contained within them.}''$^{[1]}$.
%
%\centering\includegraphics[width=\textwidth]{../figs/enclosure.pdf}
%\lfr{[1] J.G. Ram\'irez: \url{https://git.io/v5ZFh}}
%\end{ftst}
%
%%=====
%
%\begin{ftst}
%{Tolerance Intervals}
%{Definition}
%The common practice in engineering of defining specification limits by adding $\pm3\sigma$ to a given estimate of the mean arises from this definition - for a normal population $\approx 99.75\%$ of observations fall within $\mu\pm3\sigma$.
%\vone
%However, as in most cases $\sigma^2$ is unknown, we have to use $s^2$ and compensate for the uncertainty in this estimation. The two-sided tolerance interval for a given population proportion $\gamma$ is given as:$^{[2]}$
%
%\beqs
%\bar{x}\pm s\sqrt{\frac{\left(n-1\right)}{n}\frac{\left(n+z_{(\alpha/2)}^{2}\right)}{{\chi^2}_{\gamma}^{(n-1)}}}
%\eqs
%\vhalf
%\noindent wherein $\gamma$ is the proportion of the population to be enclosed, and $1-\alpha$ is the desired confidence level for the interval.
%\lfr{[2] NIST Engineering Statistics Handbook, \url{https://goo.gl/m6cxC6}}
%\end{ftst}

%=====

\begin{frame}{Wrapping up}
Statistical intervals quantify the uncertainty associated with different aspects of estimation;
\bigskip

Reporting intervals is always better than point estimates, as it provides the necessary information to quantify the location and uncertainty of your estimated values;
\bigskip

The correct interpretation is a little tricky (although not very difficult)\footnote{See the table at the end of  \url{https://git.io/v5ZFh}}, but it is essential in order to derive the correct conclusions based on the statistical interval of interest.
\end{frame}


%% TODO: Data visualization (types of graphs for indicators)
% \subsection{Visualizing indicators}
