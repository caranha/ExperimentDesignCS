\documentclass[aspectratio=169]{beamer}
\usepackage{amssymb,amsmath}
\usepackage{graphicx}
\usepackage{url}
\usepackage{color}
\usepackage{pagenote}[continuous,page]
\usepackage{cancel}   % Math "cancelto"
\usepackage{relsize}	% For \smaller
\usepackage{url}			% For \url
\usepackage{epstopdf}	% Included EPS files automatically converted to PDF to include with pdflatex

%For MindMaps
% \usepackage{tikz}%
% \usetikzlibrary{mindmap,trees,arrows}%

%%% Color Definitions %%%%%%%%%%%%%%%%%%%%%%%%%%%%%%%%%%%%%%%%%%%%%%%%%%%%%%%%%
%\definecolor{bordercol}{RGB}{40,40,40}
%\definecolor{headercol1}{RGB}{186,215,230}
%\definecolor{headercol2}{RGB}{80,80,80}
%\definecolor{headerfontcol}{RGB}{0,0,0}
%\definecolor{boxcolor}{RGB}{186,215,230}

%%% Save space in lists. Use this after the opening of the list %%%%%%%%%%%%%%%%
%\newcommand{\compresslist}{
%	\setlength{\itemsep}{1pt}
%	\setlength{\parskip}{0pt}
%	\setlength{\parsep}{0pt}
%}

%\setbeameroption{show notes on top}

% You should run 'pdflatex' TWICE, because of TOC issues.

% Rename this file.  A common temptation for first-time slide makers
% is to name it something like ``my_talk.tex'' or
% ``john_doe_talk.tex'' or even ``discrete_math_seminar_talk.tex''.
% You really won't like any of these titles the second time you give a
% talk.  Try naming your tex file something more descriptive, like
% ``riemann_hypothesis_short_proof_talk.tex''.  Even better (in case
% you recycle 99% of a talk, but still want to change a little, and
% retain copies of each), how about
% ``riemann_hypothesis_short_proof_MIT-Colloquium.2000-01-01.tex''?

\mode<presentation>
{
  \usetheme{CambridgeUS}
  \usecolortheme{dolphin}
  \useoutertheme{default}
  \useinnertheme{default}
  \setbeamercovered{invisible} % or whatever (possibly just delete it)
}
\beamertemplatenavigationsymbolsempty

\usepackage[english]{babel}
%\usepackage[latin1]{inputenc}
\usepackage{subfigure}

\usepackage{times}
\usepackage[T1]{fontenc}
\usepackage{CJKutf8}

%% makes the ppagenote command for figure references at the end.
\makepagenote
\renewcommand{\notenumintext}[1]{}
\newcommand{\ppagenote}[1]{\pagenote[Page \insertframenumber]{#1}}

\title[Experiment Design (01CH740)]{Experiment Design for Computer Sciences (01CH740)}
\author[Claus Aranha]{Claus Aranha\\{\footnotesize caranha@cs.tsukuba.ac.jp}}
\institute[U. Tsukuba]{University of Tsukuba, Department of Computer Sciences}


\subtitle[Statistical Indicators]{Topic 02 - Point and Interval Indicators}

\begin{document}
\begin{CJK}{UTF8}{ipxm}

\begin{frame}
  \maketitle

  \vfill

  \hfill \tiny{Version 2022 (Updated \today)}
\end{frame}


\section{Basic Concepts}
\begin{frame}
  \begin{center}
    Part II.a: Indicators Basic Concepts
  \end{center}
\end{frame}

\begin{frame}{Lecture Outline}

  In the first lecture, we talked about what is science, and how experiments
  (carefully designed experiments) are important in science.\bigskip

  Starting from this lecture, we will talk about how we use statistics to
  understand the data that we gather from experiments, and how we can draw
  conclusions about them.\bigskip

  Topics for today:
  \begin{itemize}
    \item \structure{Experimental Data}: Population, Observation and Sample;
    \item \structure{Point indicators}: How we obtain information about the population from samples;
    \item \structure{Interval indicators}: Indicators with quality info!
  \end{itemize}
\end{frame}

\begin{frame}{Lecture Outline}{Data and "Experiment Data"}
  When we talk about "data" in Computer Science, the first thing that comes to mind is "information that we feed to a program". {\bf For example}: images, network logs, user databases, etc.\bigskip

  In this course, we are talking about "Experiment Data", which should be understood as "Data about the result of an experiment". {\bf For example}: How long did the experiment take? What is the success rate of my program?\bigskip

  In fact, we can use the tecniques of this lecture for the first kind of data too! But to make things simpler, let's concentrate on the second kind of data.
\end{frame}

\begin{frame}{Example of Data Collection}{Using Experiment Data to characterize a system}
  \begin{columns}
    \column{0.8\textwidth}
    How can we use experiment data to learn more about the world?\bigskip

    \structure{Tsukuba University} is famous for many olympic athletes. Let's say
    I want to investigate {\bf WHY}. After thinking a bit, I come up with two questions:
    \begin{itemize}
      \item Is the olympic performance related with the body of the students?
      \item Are students in Tsukuba stronger / taller / healthier than other unis?
    \end{itemize}\bigskip

    The 1st question seems hard to answer, but maybe the 2nd is easier?\bigskip

    Imagine I take one student from Tsukuba and measure their height.
    Can this information help me answer the second question?
    \column{.2\textwidth}
    \includegraphics[width=\textwidth]{../img/irasutoya_height}
    \ppagenote{Height image from \url{https://www.irasutoya.com}}
  \end{columns}
\end{frame}

\begin{frame}{Using experiment data to characterize a system}
  From the height of only one student, I cannot really learn anything about
  the height of the students of the university {\bf in general}.
  A better approach would be to measure the height of several students.\bigskip

  \begin{center}
    \includegraphics[width=0.09\textwidth]{../img/irasutoya_height}
    \includegraphics[width=0.1\textwidth]{../img/irasutoya_height}
    \includegraphics[width=0.08\textwidth]{../img/irasutoya_height}
    \includegraphics[width=0.09\textwidth]{../img/irasutoya_height}
    \includegraphics[width=0.1\textwidth]{../img/irasutoya_height}
    \includegraphics[width=0.1\textwidth]{../img/irasutoya_height}
    \includegraphics[width=0.09\textwidth]{../img/irasutoya_height}
    \includegraphics[width=0.1\textwidth]{../img/irasutoya_height}
  \end{center}
  \bigskip

  Now that I have the height of many students, what can I say about the height
  of the students in the university, {\bf in general}?
\end{frame}

\begin{frame}{Population and Sample}
  This example introduces us to some important concepts in statistics:
  \structure{Population, Observation, and Sample}.\bigskip

  \begin{itemize}
    \item {\bf Population}: This is a set of objects that we want to learn more about, using experiments. It can be a real set (all students of the university), or a theoretical set (all possible results of running a program).

    \item {\bf Observation}: This is one element from the population. One student from the university, or one execution of the program. One data point from an experiment.\medskip

    \item {\bf Sample}: This is a set of observations. A subset of the population.
  \end{itemize}
  \bigskip

  Our goal, when we analyse data from an experiment, is to {\bf "learn something about the population, by examining the observations in the sample"}.
\end{frame}

\begin{frame}{Population and Sample}{Learn something about the population, by examining the observations in the sample}
  \begin{block}{Population}
    \begin{columns}
      \column{.18\textwidth}
      \includegraphics[width=1\textwidth]{../img/ballpool}
      \column{.65\textwidth}
      You want to know the proportion of colorful balls in a pool (you like the red ones). Because we don't know exactly how many there are, we need to {\bf make an estimation}.
    \end{columns}
  \end{block}
  \begin{block}{Sample}
    \begin{columns}
      \column{.14\textwidth}
      \includegraphics[width=1\textwidth]{../img/ballhand}
      \column{.65\textwidth}
      To learn the proportion of red balls, we pick a number of balls, and {\bf estimate that the proportion of red balls in the pool is equal to the proportion in my hand}.
    \end{columns}
  \end{block}
\end{frame}

\begin{frame}{Population, Model and Parameters}{What is a model?}
  A model is a description of the population, focused on the scientific questions that we want to make.\bigskip

  \begin{itemize}
    \item The balls are distributed evenly in the pool, so I can take my sample from any part.
    \item The height of the students in the university can be represented by a {\bf normal curve}, with mean $\mu$ and standard deviation $sd$.
    \item The SIR infection model says that {\bf susceptible} people become {\bf infected}, and then {\bf Recovering}, so if we can learn the \structure{number of people in each group} and the \structure{transition probability}, we can predict the progress of a disease.
  \end{itemize}\bigskip

  The goal of many experiments is to use data to {\bf estimate the parameters of a model}.
\end{frame}

\begin{frame}{Population, Model and Parameters}{Example of model parameters}
  \begin{columns}[T]
    \column{0.5\textwidth}
      By analysing the sample data obtained from an experiment, we can estimate
      values for the parameters of the model.\bigskip

      The {\bf true value} of the parameters in the model (what we don't know) is usually called $\theta$. The values \structure{estimated from the experiment data} is named $\hat{\theta}$. Note the difference!\bigskip

      The {\bf model} is determined during the experiment design phase.\medskip

      {\smaller
      \emph{Every model is wrong, but some models are useful!}
      }
    \column{0.5\textwidth}
    \includegraphics[width=.8\textwidth]{../img/wikipedia_triceratops}
    \ppagenote{Triceratops information table CC by Zachi Evenor and MathKnight}
  \end{columns}
\end{frame}



\begin{frame}{Sample Data, Statistics, and Parameters}{Statistics are functions on the data}
  \begin{block}{}
    A \structure{statistic} (singular) is a function, that takes experiment data as its input.
  \end{block}\medskip

  We estimate the value of the parameter of a model, by calculating a statistic, based on data we obtained from the sample. For example:\bigskip

  \begin{itemize}
    \item How long does it take to run a program? Run the program many times (sample), and calculate the \structure{mean execution time} (statistic).
    \item How effective is a drug? We give the drug to sick patients (sample), and count \structure{how~many~patients} get better after two days (statistic);
    \item Is it better to add more perceptrons to a neural network? We run the network with different sizes (sample), and calculate the \structure{correlation between size and accuracy} (statistic).
  \end{itemize}\bigskip
\end{frame}

\begin{frame}{Point and Interval Indicators}
  In this lecture, we focus on two types of statistics: {\bf Point Estimators} and
  {\bf Interval Estimators}.\bigskip

  \begin{itemize}
    \item {\bf Point Estimators}: estimate \structure{one value} for a parameter from a sample;

    \item {\bf Interval Estimators}: estimate a \structure{range of values} for a parameter;
  \end{itemize}
\end{frame}

\begin{frame}{Statistics are Random Variables}{The estimated value depends on the population and the sample}

  \begin{itemize}
    \item The value of an \structure{statistic} is calculated from the \structure{sample}, that we obtained from the experiment.
    \item The \structure{sample}, in turn, is a subset of the \structure{population}, which is what we want to study.\bigskip

    \item The implication is that {\bf every time we do a new experiment, the value calculated by an statistic will be a little bit different}.
  \end{itemize}\vfill

  \begin{exampleblock}{Don't Worry!}
    These differences are expected and, if your experiment design is good, can be controlled. Just remember that a value you calculate from a sample is not necessarily {\bf the truth}$^{tm}$
  \end{exampleblock}


\end{frame}

\begin{frame}{Statistics are Random Variables}{What is measured by statistical tests?}
  \begin{columns}[T]
    \column{.3\textwidth}
      \includegraphics[width=1\textwidth]{../img/sampling_distribution}
    \column{.7\textwidth}
    A statistic can show different values, depending on the sample obtained.
    So it is useful to treat statistics as \structure{Random~Variables}.\bigskip

    As a Random Variable, a statistic has a {\bf sampling distribution}.
    The sampling distribution is a model that describes what values we expect
    to obtain from a statistic, when we perform an experiment.\bigskip

    In general, statistical tests work by estimating the sampling distribution,
    and using the sampling distribution to understand the population under study.
    (We will study statistical tests next week).
  \end{columns}

\end{frame}



%%%%%%%%%%%%%%%%%%%%%%%%%%%%%%%%%%%%%%%%%%%%%%%%%%%%%%%%%%%%%%%%%

\section{Point Estimators}
\begin{frame}
  \begin{center}
    Part II.b - Point Estimators
  \end{center}
\end{frame}

\subsection{Definition}

\begin{frame}{Definition of a Point Indicator}

A more formal definition of \structure{Point Indicator} is: \emph{a statistic
that provides the value of maximum plausibility for an (unknown) population
parameter $\theta$}.\bigskip

In other words, a P.I. is a function that, given a sample obtained in an
experiment, calculates the {\bf more plausible} value for a parameter $\theta$
that describes the population of interest.\bigskip

\begin{itemize}
  \item Consider a random variable $X$, and a parameter $\theta$ that describes its distribution.
  \item Consider a sample $x = \{x_1, x_2, \ldots, x_n\}$ obtained from $X|\theta$
  \item The function $\hat{\Theta} = h\left(x\right)$ is a \structure{point estimator} of $\theta$, and the value $\hat\theta$ calulated from this function is a \structure{point estimate} of $\theta$.
\end{itemize}
\end{frame}

\begin{frame}{Examples}
Point estimation problems arise frequently in all areas of science and engineering, whenever there is a need for estimating, a population parameter. For example:\bigskip

\begin{itemize}
  \item the population mean, $\mu$;
	\item the population variance, $\sigma^2$;
	\item the population proportion, $p$;
	\item the difference in the means of two populations, $\mu_1-\mu_2$;
	\item etc...
\end{itemize}\bigskip

{\bf Important!} For each case, there are multiple statistics that can estimate a parameter from the sample (multiple functions). Which estimator should be used, depends on the mathematical properties of the statistic.
\end{frame}

\subsection{Statistical Errors and Biases}
\begin{frame}{Statistical Errors and Biases}
  We said before that the value calculated by a point estimator (\structure{the estimate, $\hat\theta$}) depends on the sample. So if we are unlucky with our sample, the estimated value can be very different from the true parameter value, $\theta$.\bigskip

  Talking about this difference, we want to consider {\bf Statistical Error} and {\bf Statistical Bias}:

  \begin{itemize}
    \item {\bf Error}: The difference between an estimate and the true value of a parameter;
    \item {\bf Bias}: The property of a statistic that \structure{systematically} produces wrong estimates;
  \end{itemize}\vfill

  It is important to note that besides \structure{Statistical Bias}, we also have \structure{Structural Bias}, which depends a lot on the design of the experiment. Today, we will focus on the first one.
\end{frame}


\begin{frame}{Statistical Errors and Biases}{Example of a biased statistic}
  Imagine that we want to know the typical height of the students of the CS Program.\bigskip

  Our experiment design is to get a random sample of 10 students, measure their height, and use a statistic to calculate the point estimate of the height, from the sample data. Let's consider two statistics:
  \begin{itemize}
    \item The point estimate of the height is {\bf the minimum height} from the sample.
    \item The p. e. of the height is {\bf the height of a random observation} from the sample;
  \end{itemize}\bigskip

  The first statistic will give you a result that is {\bf usually smaller} than what we could consider a representative value for the students' height.\bigskip

  \alert{To think about}: What could we say (mathematically) about the second statistic?
\end{frame}

\begin{frame}{Unbiased estimators}
\begin{itemize}
  \item A good estimator should consistently generate estimates that are close to the real value of the parameter $\theta$.
  \item An \structure{unbiased} estimator will generate estimates where the errors, when they happen, are equally distributed above and below the real value of $\theta$.
\end{itemize}\bigskip

Mathematically, we say that an estimator $\hat{\Theta}$ is said to be \textit{unbiased} for parameter $\theta$ if its \structure{expected value} is $\theta$ ($E\left[\hat\Theta\right] = \theta$). or, equivalently:\bigskip

\begin{equation*}
E\left[\hat{\Theta}\right] - \theta = 0
\end{equation*}\bigskip

The difference $E\left[\hat{\Theta}\right] - \theta$ is referred to as the \textit{bias} of a given estimator.
\end{frame}

\begin{frame}{Unbiased estimators}{Example: sample's average}
Let's show briefly that the average of a sample of values is an unbiased estimator for the mean of the population:\bigskip

Let $x_1,\ldots,x_n$ be a random sample from a given population $X$, and $\bar{x}$ be the average of the sample. The {\bf value of interest} in this population is modeled by a distribution with mean $\mu$.\bigskip

Noting that the expected value of one observation $x_i$ is the mean $\mu$ of the population, the expected value of the average of the sample will be:\bigskip

\begin{equation*}
E\left[\bar{x}\right] = E\left[\frac{1}{n}\sum\limits_{i=1}^{n}x_i\right] =
\frac{1}{n}\sum\limits_{i=1}^{n}E\left[x_i\right] =
\frac{1}{n}\sum\limits_{i=1}^{n}\mu = \mu
\end{equation*}
\vfill
\end{frame}

\begin{frame}{Unbiased estimators}{If statistical bias is the same, what else is important for a statistic?}

It is usually possible to define multiple unbiased estimators for a parameter $\theta$. However, the \structure{variance} of these estimators may be different.\bigskip

\begin{columns}[T]
    \column{0.5\textwidth}\includegraphics[width=1\textwidth]{../img/unbiased_variance.png}
    \column{0.5\textwidth}
    For example, the \structure{average value of a sample} and the \structure{value of a single observation} are both unbiased estimators of the mean. The first one has smaller variance.\bigskip

    We usually want to obtain the \textit{minimal-variance unbiased estimator} (MVUE).
\end{columns}\bigskip
This is because they will usually generate estimates $\hat\theta$ that are closer to the real value of $\theta$.

\ppagenote{Image: D.C.Montgomery,G.C. Runger, \textit{Applied Statistics and Probability for Engineers},Wiley 2003.}
\end{frame}

\subsection{Statistics about Statistics}
\begin{frame}{Standard error of a point estimator}
In the last slide we showed that a point estimator $\hat\Theta$ will have an associated distribution, with its mean ($\mu_{\hat\Theta}$) and variance ($Var\left[\hat{\Theta}\right]$).\bigskip

So we can talk about the \structure{standard error} of an estimator $\hat{\Theta}$ as:
\begin{equation*}
\sigma_{\hat{\Theta}} = \sqrt{Var\left[\hat{\Theta}\right]}
\end{equation*}\bigskip

Because the values of the estimator depend on the parameters of the population, we cannot know precisely the value of $\sigma_{\hat\Theta}$ or $\mu_{\hat\Theta}$. But we can {\bf estimate the error of the estimator}, using data from the sample. In this case, we refer to it as the \textit{estimated standard error}, $\hat{\sigma}_{\hat{\Theta}}$ (the notations $s_{\hat{\Theta}}$ and $se(\hat{\Theta})$ are also common).

\hfill\includegraphics[width=.1\textwidth]{../img/yodawg}
\end{frame}

%% TODO: this part was taken directly from campelo's materials, Needs update?
\begin{frame}{Standard error of a point estimator}{Examples}
  Assuming a random variable $X$ under a gaussian distribution, and a sample error $s$, we can calculate the standard errors of several common point indicators\footnote{See Ahn and Fessler (2003), \textit{Standard Errors of Mean, Variance, and Standard Deviation Estimators}: \url{https://git.io/v5Z5v}}

\begin{equation*}
\hat{\sigma}_{\bar{X}} = \frac{s}{\sqrt{n}}
\end{equation*}

\begin{equation*}
\hat{\sigma}_{S^2} = s^2\sqrt{\frac{2}{n-1}}
\end{equation*}

\begin{equation*}
\hat{\sigma}_{S} = \frac{s}{\sqrt{2(n-1)}} + O\left(\frac{1}{n\sqrt{n}}\right)\approx \frac{s}{\sqrt{2(n-1)}}
\end{equation*}\bigskip
\end{frame}

\subsection{Example}
\begin{frame}{Point Estimator Use Case}
  \begin{columns}[T]
    \column{0.4\textwidth}
    \hspace{.05\textwidth}\includegraphics[width=.9\textwidth]{../img/pixabay_cable}
    \ppagenote{Coaxial cable image from \url{https://pixabay.com}}

    \column{0.6\textwidth}
    Consider a factory that produces coaxial cables.\bigskip

    Let's assume that the resistance values of the cables produced by this factory
    are distributed following a normal distribution, with mean $50\Omega$ and variance $4\Omega$.\bigskip

    $X\sim\mathcal{N}\left(\mu=50,\sigma^2=4\right)$
\end{columns}
\end{frame}

\begin{frame}{Point Estimator Use Case}
  \begin{columns}[T]
    \column{.4\textwidth}
    \hspace{.05\textwidth}\includegraphics[width=.9\textwidth]{../img/pixabay_cable}
    \column{.6\textwidth}
    Let's imagine an experiment, where we take a random sample of $25$ cables
    from the production process (to see if the resistance of the cables is as
    expected).\bigskip

    The \structure{sample mean} of the the observations is:
    \begin{equation*}
    \bar{x} = \frac{1}{25}\sum\limits_{i=1}^{25}{x_i}
    \end{equation*}
  \end{columns}
  \bigskip



The \structure{sample mean} follows a normal distribution, with $E[\bar{x}] = \mu = 50\Omega$ and $\sigma_{\bar{x}} = \sqrt{\sigma^2/25} = 0.4\Omega$.\bigskip

Note that the error of the estimator depends on the sample size. So if we collect a bigger or smaller sample, the error will decrease or increase, respectively.
\end{frame}

%% TODO: add one slide with examples of estimating the mean resistance of
%% the cables, with smaller and bigger sample sizes.

\subsection{The Central Limit Theorem}

%% TODO: this part was taken directly from campelo's materials, Needs update.
\begin{frame}{The Central Limit Theorem}{We love Normal distributions.}

In the previous example, the resistance of cables in our imaginary factory follows a normal distribution. \bigskip

So the \structure{sample average} statistic for this model will also follow a normal distribution.\bigskip

Normal distributions have some very nice properties (that we will explore in the next class), so it is nice when our statistics follow them.\bigskip

Does this only happen when the population ("real") distribution is real?\bigskip
\end{frame}

\begin{frame}{The Central Limit Theorem}{Sample-based statistics follow a normal distribution under certain conditions}

The {\bf Central Limit Theorem} states that, under certain conditions, the sampling distribution of the mean will tend to be approximately normal, even for populations with arbitrary distributions.
\bigskip

More formally, let $x_1,\ldots,x_n$ be a sequence of \textbf{independent and identically distributed} (\textbf{iid}) random variables, with mean $\mu$ and finite variance $\sigma^2$. We define the statistic $z_n$ as:
\begin{equation*}
z_n = \frac{\sum\limits_{i=1}^{n}{(x_i)} - n\mu}{\sqrt{n\sigma^2}} = \frac{\bar{x} - \mu}{\sqrt{\sigma^2/n}}
\end{equation*}

Under these conditions, $z_n$ is distributed asymptotically as a standard Normal variable, that is, $z_n\sim\mathcal{N}(0,1)$.
\end{frame}

\begin{frame}{The Central Limit Theorem}

The {\bf Central Limit Theorem} is one of the most useful properties for statistical inference. The CLT allows the use of techniques based on the Normal distribution, even when the population under study is not normal.\bigskip

For ``well-behaved'' distributions (continuous, symmetrical, unimodal - the usual bell-shaped pdf we all know and love) even small sample sizes are commonly enough to justify invoking the CLT and using techniques that assume a model with a normal distribution.\bigskip

For more details on the CLT, see \url{https://www.encyclopediaofmath.org/index.php/Central_limit_theorem}
\end{frame}

%% TODO: Set up my own CLT code example

% \begin{frame}
% {Sampling Distributions}
% {The Central Limit Theorem}
% For an interactive demonstration of the CLT, download the files in {\small\url{https://git.io/vnPj8}} and run on RStudio.
% \bigskip
%
% {\centering\includegraphics[width=\textwidth]{../img/CLTdemo.png}}
% \end{frame}
%
% \section{Interval Estimators}
% \section{Basic Concepts}
% \begin{frame}
%   \begin{center}
%     Indicators, Part III - Interval Estimators
%   \end{center}
% \end{frame}

%% TODO: This part is too technical. Add a concrete example of the Confidence Interval, with actual numbers, when explaining what it is.
\section{Interval Indicators}
\begin{frame}
  \begin{center}
    Part II.c - Interval Estimators
  \end{center}
\end{frame}

\begin{frame}{Statistical Intervals}{How sure are you?}
As we saw, point indicators are statistics (functions) that estimate values of a population parameter, from data in the sample.\bigskip

\structure{Statistical intervals} are functions that quantify {\bf the uncertainty} associated to an estimate;
\bigskip

Let's remember the coaxial cable factory example:
\begin{itemize}
  \item A cable factory produces cable, and we are interested in their resistance.
  \item The cable producion can be described as a normal distribution, with $\mu = 50, \sigma = 2$.
\end{itemize}\bigskip

Let's now suppose that we obtained a sample with $n=25$ observations, and calculated a sample mean of $\bar{x} = 48$. Because of the sample variability, it is likely that this estimate is not the true value of $\mu$, but how much error can we assume exist in this statistic?
\end{frame}

\begin{frame}{Statistical Intervals}{Definition}
A {\bf statistical interval} defines a region that is\structure{likely to contain the true value of an estimated parameter}.
\bigskip

The statistic interval allows us to quantify the {\bf level of uncertainty} associated with the estimation. This information helps us arrive at sound conclusions about the parameter, at {\bf predefined} levels of certainty.
\bigskip

Three of the most common types of interval are:

\begin{itemize}
  \item Confidence Intervals;
  \item Tolerance Intervals;
  \item Prediction Intervals;
\end{itemize}
\end{frame}


%=====

\begin{frame}{What is a Confidence Interval?}
Confidence intervals quantify the degree of uncertainty associated with the estimation of population parameters such as the mean or the variance.\bigskip

Can be defined as ``\textit{the interval that contains the true value of a given population parameter with a confidence level of $100(1-\alpha)$}'';\bigskip

Another useful definition is to think about confidence intervals in terms of confidence \textit{in the method}: ``The method used to derive the interval has a hit rate of $95\%$'' - i.e., the interval generated has a $95\%$ chance of `capturing' the true population parameter.''
\end{frame}

%=====



%% TODO: the explanation of $z$ and $t$ comes out of nowhere in the definition of CI. This part needs to be broken down a bit more.

\begin{frame}{CI on the Mean of a Normal Variable}
The two-sided $CI_{(1-\alpha)}$ for the mean of a normal population with known variance $\sigma^2$ is given by:
\begin{equation*}
\bar{x}+z_{\alpha/2}\frac{\sigma}{\sqrt{n}}\leq\mu\leq\bar{x}+z_{1-\alpha/2}\frac{\sigma}{\sqrt{n}}
\end{equation*}
\noindent where $(1-\alpha)$ is the confidence level and $z_{x}$ is the $x$-quantile of the standard normal distribution.
\bigskip

For the more usual case with an unknown variance,
\begin{equation*}
\bar{x}+t_{\alpha/2}^{(n-1)}\frac{s}{\sqrt{n}}\leq\mu\leq\bar{x}+t_{1-\alpha/2}^{(n-1)}\frac{s}{\sqrt{n}}
\end{equation*}
\noindent where $t_{x}^{(n-1)}$ is the $x$-quantile of the t distribution with $n-1$ degrees of freedom.
\end{frame}

\begin{frame}
{Confidence Intervals}
{Example: 100 $CI_{.95}$ for a sample of 25 observations}
\centering\includegraphics[width=.8\textwidth]{../img/CIs.pdf}
\footnote{For an interactive demonstration of the factors involved in the definition of a confidence interval, Run the files at \url{https://git.io/vxXGj} on RStudio.}
\end{frame}


%=====

\begin{frame}{CI on the Variance and Standard Deviation of a Normal Variable}
A two-sided confidence interval on the variance of a normal variable can be easily calculated:
\begin{equation*}
\frac{(n-1)s^2}{{\chi^2}_{1 - \alpha/2}^{(n-1)}}\leq\sigma^2\leq\frac{(n-1)s^2}{{\chi^2}_{\alpha/2}^{(n-1)}}
\end{equation*}
\noindent where ${\chi^2}_{x}^{(n-1)}$ represents the x-quantile of the $\chi^2$ distribution with $n-1$ degrees of freedom. For the standard deviation one simply needs to take the squared root of the confidence limits.
\end{frame}

%=====

%\begin{ftst}
%	{Bootstrap Confidence Intervals}
%	{Using resampling}
%	Confidence intervals can also be constructed using a resampling technique called \textit{bootstrap};
%	\vone
%	This method works by resampling (with replacement) from the
%
%\end{ftst}

%\begin{ftst}
%{Prediction Intervals}
%{Definition}
%Prediction intervals quantify the uncertainty associated with forecasting the value of a future observation;
%\vone
%Essentially, one is interested in obtaining an interval within which he or she can declare that the next observation will fall with a given probability;
%\vone
%For a normal distribution, the tolerance interval for a single next observation (given an existing sample of size $n$) is:
%\beqs
%\bar{x}+t_{\alpha/2}^{(n-1)}s\sqrt{1 + \frac{1}{n}}\leq x_{n+1}\leq\bar{x}+t_{1-\alpha/2}^{(n-1)}s\sqrt{1 + \frac{1}{n}}
%\eqs
%\end{ftst}
%
%%=====
%
%\begin{ftst}
%{Tolerance Intervals}
%{Definition}
%``\textit{A tolerance interval is an \textbf{enclosure} interval for a specified proportion of the sampled population, not its mean or standard deviation. For a specified confidence level, you may want to determine lower and upper bounds such that a given percent of
%the population is contained within them.}''$^{[1]}$.
%
%\centering\includegraphics[width=\textwidth]{../figs/enclosure.pdf}
%\lfr{[1] J.G. Ram\'irez: \url{https://git.io/v5ZFh}}
%\end{ftst}
%
%%=====
%
%\begin{ftst}
%{Tolerance Intervals}
%{Definition}
%The common practice in engineering of defining specification limits by adding $\pm3\sigma$ to a given estimate of the mean arises from this definition - for a normal population $\approx 99.75\%$ of observations fall within $\mu\pm3\sigma$.
%\vone
%However, as in most cases $\sigma^2$ is unknown, we have to use $s^2$ and compensate for the uncertainty in this estimation. The two-sided tolerance interval for a given population proportion $\gamma$ is given as:$^{[2]}$
%
%\beqs
%\bar{x}\pm s\sqrt{\frac{\left(n-1\right)}{n}\frac{\left(n+z_{(\alpha/2)}^{2}\right)}{{\chi^2}_{\gamma}^{(n-1)}}}
%\eqs
%\vhalf
%\noindent wherein $\gamma$ is the proportion of the population to be enclosed, and $1-\alpha$ is the desired confidence level for the interval.
%\lfr{[2] NIST Engineering Statistics Handbook, \url{https://goo.gl/m6cxC6}}
%\end{ftst}

%=====

\begin{frame}{Wrapping up}
Statistical intervals quantify the uncertainty associated with different aspects of estimation;
\bigskip

Reporting intervals is always better than point estimates, as it provides the necessary information to quantify the location and uncertainty of your estimated values;
\bigskip

The correct interpretation is a little tricky (although not very difficult)\footnote{See the table at the end of  \url{https://git.io/v5ZFh}}, but it is essential in order to derive the correct conclusions based on the statistical interval of interest.
\end{frame}


%% TODO: Data visualization (types of graphs for indicators)
% \subsection{Visualizing indicators}



\begin{frame}{Recommended Reading}
  \begin{itemize}
    \item \emph{D.W. Stockburger}, The Sampling Distribution. In: Introductory Statistics: Concepts, Models, and Applications -
    \url{http://psychstat3.missouristate.edu/Documents/IntroBook3/sbk17.htm}
    \item \emph{J.G. Ramírez}, Statistical Intervals: Confidence, Prediction, Enclosure: \url{https://git.io/v5ZFh}
    \item Crash Course Statistics Playlist, in particular videos \#3 to \#7: \url{https://www.youtube.com/playlist?list=PL8dPuuaLjXtNM_Y-bUAhblSAdWRnmBUcr}
  \end{itemize}
\end{frame}

\begin{frame}{Programming in R}
  The material for this week includes some coding examples. These examples are
  written in the {\bf R} language.\bigskip

  Although we will have an R tutorial in the future, you can read the
  following material to get yourself acquainted with R:\bigskip

  \begin{itemize}
    \item R for beginners: \url{https://cran.r-project.org/doc/contrib/Paradis-rdebuts_en.pdf}
    \item Rstudio: \url{https://rstudio.com}
  \end{itemize}
\end{frame}

%%%%%%%%%%%%%%%%%%%%%%%%%%%%%%%%%%%%%%%%%%%%%%%%%%%%
\section{Backmatter}
\begin{frame}{About these Slides}
  These slides were made by Claus Aranha, 2020. You are welcome to copy, re-use and modify this material.
  \bigskip

  These slides are a modification of "Design and Analysis of Experiments (2018)" by Felipe Campelo, used with permission.
  \bigskip

  Individual images in some slides might have been made by other
  authors. Please see the references in each slide for those cases.
\end{frame}

\begin{frame}[allowframebreaks]{Image Credits}
  \printnotes
\end{frame}

\end{CJK}
\end{document}
