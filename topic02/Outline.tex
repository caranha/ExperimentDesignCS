\documentclass{beamer}

\usepackage{amssymb,amsmath}
\usepackage{graphicx}
\usepackage{url}
\usepackage{color}
\usepackage{pagenote}[continuous,page]
\usepackage{cancel}   % Math "cancelto"
\usepackage{relsize}	% For \smaller
\usepackage{url}			% For \url
\usepackage{epstopdf}	% Included EPS files automatically converted to PDF to include with pdflatex

%For MindMaps
% \usepackage{tikz}%
% \usetikzlibrary{mindmap,trees,arrows}%

%%% Color Definitions %%%%%%%%%%%%%%%%%%%%%%%%%%%%%%%%%%%%%%%%%%%%%%%%%%%%%%%%%
%\definecolor{bordercol}{RGB}{40,40,40}
%\definecolor{headercol1}{RGB}{186,215,230}
%\definecolor{headercol2}{RGB}{80,80,80}
%\definecolor{headerfontcol}{RGB}{0,0,0}
%\definecolor{boxcolor}{RGB}{186,215,230}

%%% Save space in lists. Use this after the opening of the list %%%%%%%%%%%%%%%%
%\newcommand{\compresslist}{
%	\setlength{\itemsep}{1pt}
%	\setlength{\parskip}{0pt}
%	\setlength{\parsep}{0pt}
%}

%\setbeameroption{show notes on top}

% You should run 'pdflatex' TWICE, because of TOC issues.

% Rename this file.  A common temptation for first-time slide makers
% is to name it something like ``my_talk.tex'' or
% ``john_doe_talk.tex'' or even ``discrete_math_seminar_talk.tex''.
% You really won't like any of these titles the second time you give a
% talk.  Try naming your tex file something more descriptive, like
% ``riemann_hypothesis_short_proof_talk.tex''.  Even better (in case
% you recycle 99% of a talk, but still want to change a little, and
% retain copies of each), how about
% ``riemann_hypothesis_short_proof_MIT-Colloquium.2000-01-01.tex''?

\mode<presentation>
{
  \usetheme{CambridgeUS}
  \usecolortheme{dolphin}
  \useoutertheme{default}
  \useinnertheme{default}
  \setbeamercovered{invisible} % or whatever (possibly just delete it)
}
\beamertemplatenavigationsymbolsempty

\usepackage[english]{babel}
%\usepackage[latin1]{inputenc}
\usepackage{subfigure}

\usepackage{times}
\usepackage[T1]{fontenc}
\usepackage{CJKutf8}

%% makes the ppagenote command for figure references at the end.
\makepagenote
\renewcommand{\notenumintext}[1]{}
\newcommand{\ppagenote}[1]{\pagenote[Page \insertframenumber]{#1}}

\title[Experiment Design (01CH740)]{Experiment Design for Computer Sciences (01CH740)}
\author[Claus Aranha]{Claus Aranha\\{\footnotesize caranha@cs.tsukuba.ac.jp}}
\institute[U. Tsukuba]{University of Tsukuba, Department of Computer Sciences}


\title[]{Experiment Design}
\subtitle[]{Lecture 1: Introduction\footnote{version 1.2}}
\author[Claus Aranha]{Claus Aranha\\{\footnotesize caranha@cs.tsukuba.ac.jp}}
\institute{Department of Computer Science}

\begin{document}

\begin{frame}
  \tableofcontents
\end{frame}

\begin{frame}
  \frametitle{Class Outline}
  \begin{itemize}
  \item \structure{Ask how is everyone today:} Talk to them about Science
    Popularization (Startalk, Numberphile, Cosmos);
  \item \structure{Lecture Dates:} No class next week (29th). In the
    other week, the class will be on Friday (9th);
  \item \structure{Homework:} Show of hands for homework. Ask for a
    few people to introduce their research, ask about experiments --
    how do you validate your research? Write on the blackboard ways
    for validating their research;
  \item \structure{Examination:} Remind them that the final evaluation
    will be based on a work where they have to make an experiment
    design in the context of their own research;
  \end{itemize}
\end{frame}

\begin{frame}
  \frametitle{Class Outline}
  \begin{itemize}
  \item \structure{Define the Class Outline for Today:} Today we will
    be describing how to prepare an experiment in computer
    sciences. Our focus will be the idea of \structure{New
      Experimentalism}, a series of best practices for analysing
    scientific ideas based on experimental findings.
  \item \structure{Class resource:} A very large part of today's class
    is based off the second and third chapter to ``Experimental
    Methods for the Analysis of Optimization Algorithms'', by Thomas
    Bartz-Bielstein et al. Read it for more info!
  \item \structure{Review the last class briefly}: The conclusion is
    that today we will talk about good practices when preparing a
    scientific activity.
  \end{itemize}
\end{frame}


\begin{frame}
  \frametitle{Experimentalism}
  \begin{itemize}
    \item Experimentalism: Learning from the observation of data that
      arises from experiments.
    \item It is very common to do this in a hapzard manner, but it is
      important to give a good structure to the experiment. A well
      structured experiment gives results that reliable, and we can
      try to specify what kind of results that we want.
    \item Definition of Factors (Factors are quantities that change in
      an experiment;
    \item Computer Scientists (unlike say, agriculture) can control
      most, if not all, factors in an experiment; Therefore, it is
      possible to make much more meaningful experiments -- and also
      much more meaningless experiments as well.
  \end{itemize}
\end{frame}

\begin{frame}
  \frametitle{Experimentation in computer science}
  \begin{itemize}
  \item \structure{The good old days}: In the good old days, running
    two algorithms and comparing the averages was enough. However,
    we have almost total control over our experiment environment, so
    we can do much better than that.
  \item One point is the use of better statistics (we will talk
    about that later).
  \item But before we even get to discuss better statistics, we need
    to think about how we approach the decision of making an
    experiment.
  \item This is the idea of experiment design.
  \end{itemize}
\end{frame}

\begin{frame}
  \frametitle{Characteristics of Computer Science experiments}
  \begin{itemize}
  \item Before anything else, we need to ask \structure{Why are we doing this experiment?}
  \item These are frequent reasons for performing experiments in Computer Science:
    \begin{itemize}
    \item Are we trying to make our algorithm solve a problem that was
      too hard before? Are we trying to solve a harder version of an
      old problem, or a new problem that has not been solved before?
    \item Are we trying to improve the \structure{performance}
      compared to the current best algorithm? Maybe making it faster,
      or use less memory?
    \item Are we trying to understand WHY the algorithm works? What
      makes it work faster or slower? What are the factors that have
      the biggest influence in the correctness or performance of an
      algorithm?
    \end{itemize}
  \item When we define \structure{why} we want to do an experiment,
    we can decide \structure{what} the best experiment is. (some
    people use the same experiment for everything)
  \end{itemize}
\end{frame}

\begin{frame}
  \frametitle{Pre-experimental Setup}
  (Improving the performance experiments)
  \begin{itemize}
  \item Understanding the goals of the experiment will allow you to
    determine the kind of experimentation to make, and what factors to
    isolate during experimentation.
  \item Should we focus on Algorithm Parameters? (Algorithm tuning) -
    Population size, number of layers in a perceptron model, number of
    nodes in a parallel environment. Can you give other ideas?
  \item Should we focus on Problem Parameters? (Problem design) - Data
    set to be used, number of dimensions in an artificial data
    set. Number of constraints to be taken into account, or to be
    ignored.
  \item Use of real world problems vs Theoretical problems. Real world
    settings usually allow for little variation in the Problem-related
    factors, while theoretical problem settings allow for a huge
    control of factors.
  \end{itemize}
\end{frame}

\begin{frame}
  \frametitle{Pre-experimental Design}
  (Understanding the algorithm-kind of experiments)
  \begin{itemize}
  \item We are more interested in the effect of changing the factors
    in the outcome, than the outcome itself.
  \item Because of this, we need to stablish a ``policy'' for varying
    the parameters in a controlled fashion.
  \item \structure{Uncertainty Analysis} - for a fixed set of factors,
    what is the expected variation in the performance/behavior of an
    experiment?
  \item \structure{Sensitivity Analysis} - now that we know the
    variance, which are the factors that contribute most to this
    variance?
  \item \structure{Structure of The experiment} - Observation of the
    System vs Planned Interference;
  \end{itemize}
\end{frame}

\begin{frame}
  \frametitle{Experimentation Models}
  \begin{itemize}
  \item In the context of experimentalism, the idea of establishing
    \structure{Scientific Models} is crucial for defining the best
    experiments to be conducted;
  \item Three levels of scientific models: \structure{Models of
    Scientific Hypothesis, Models of Experiment, Models of Data};
  \item \structure{Scientific Models} are used to break a scientific
    inquiry into one or more ``decision points'' that can be analysed
    in isolation.
  \item \structure{Experimental Models} transform the scientific
    models into practical questions that can be answered by using
    experiments;
  \item \structure{Data Models} are used to generate and model the
    data obtained FROM the experiments, from which the above models
    are judged.
  \item Show the example with the image and the tree of models;
  \end{itemize}
\end{frame}

\begin{frame}
  \frametitle{Experimentation Issues}
  \begin{itemize}
  \item \structure{Floor and Ceiling effects}: problem is too easy, problem is too
    hard, N is too big, N is too small;
  \item \structure{Confounded Effects}: Factors are not perfectly
    isolated, We cannot change all factors at the same time. 
  \item If we are not sure how the factors interact (epistemologic
    effect), we need to use some sort of factorial design (Like the
    Latin Hypercube Design);
  \item How to try and control factors -- See Slide: Experimental
    Design (3) (Repetition of experiments, variation of the order, etc)
  \item Factor control: Expected factors (algorithm, problem),
    Possible factors (Software version, operating system,
    multi-tasking), unexpected factors (weather, color of the
    experimenter socks)
  \item Sometimes, we should blow a trumpet to tulips every morning
    for a month. Probably nothing would happen, but what if it did?
  \end{itemize}
\end{frame}

%% Statystics analysis
\section{Statistics Summary}
\subsection{Statistics}
\begin{frame}
  \frametitle{Proper Statistical Analysis!}
  \begin{itemize}
  \item Follow the Appendix;
  \item The goal right now is not to get all the calculations right
    (we will do that in a future class), but to get the concept of
    statistical significance across.
  \item Definition of random variable -- the result of an experiment
    is a random variable, if the experiment is not deterministic.
  \item The distribution of a random variable is the list of all
    possible values for all possible probabilities.
  \item Definition of Population and Sample, and how the mean of both
    can be different.
  \end{itemize}
\end{frame}

\begin{frame}
  \frametitle{Statistical Significance Analysis Outline 2}
  \begin{itemize}
    \item Show that as the sample changes, values such as the mean of
      the sample and the variance of the sample are also Random
      Variables;
    \item Random variables are not necessarily random;
    \item Therefore, if we repeat the experiment, we can have results
      that confirm or that deny or hypothesis, or that are very
      close. How can we know the validity of one particular result?
    \item In other words, we are asking \structure{how can we know if
      a good result is consistently good, in the face of random
      factors?}
  \end{itemize}
\end{frame}

\begin{frame}
  \frametitle{Statistical Analysis Outline 3}
  \begin{itemize}
  \item Let us give a very simple example of how this is done. 
  \item Again, the statistic formulas will be made explicit in the
    following classes. Today I just want you to understant the
    principle of statistical testing.
  \item We take a sample of a random variable (our experimental
    result), from this sample, we can calculate the means and the
    variance. If we assume that this random variable has a certain
    distribution, we can know the probability for many different
    values.
  \item For now, let's assume a Normal distribution. This is not
    always the best assumption, but it is a good one for our purposes
    (the assumed distribution of our random variable is a factor of
    the experiment!).
  \end{itemize}
\end{frame}

\begin{frame}
  \frametitle{Statistical Analysis Outline 4}
  \begin{itemize}
  \item Now that we have a distribution for our random variable, we
    define the hypothesis that we want to test, we do this by defining
    a \structure{null hypothesis} and a \structure{alternate
      hypothesis}.
  \item This hypothesis can be defined as a fixed value, or as a
    second random variable. And we can define it by relations of
    equality or inequality. \structure{Some Examples}. 
  \item This allows us to decide to ``reject'' the null hypothesis or
    ``accept'' the alternate hypothesis. Talk about Type I and Type II
    errors, remember that these values are a sample of a random
    variable!
  \item We can also think about minimum and maximum values for the
    null and alternate hypothesis, in order to minimize the
    probability of the errors.
  \item There are many formulas for this, but today I just want you
    to have a ``intuition'' for statistical testing. You will be
    sick of it by the end of the course.
  \end{itemize}
\end{frame}

\begin{frame}
  \frametitle{Statistical Analysis Outline 5}
  \begin{itemize}
  \item All the statistical analysis have assumptions about the data
    -- we need to \structure{validate} these assumptions.
  \item For example, assumption of normalty
  \item Not only that, but in some cases a statistically significant
    difference can be very small, if we have $N$ very big. We need to
    calculate the scientific relevance of the result.
  \item We will talk about all of this in the next class.
  \end{itemize}
\end{frame}

%%% End Of Class

\section{endofClass}
\subsection{endofClass}
\begin{frame}
  \frametitle{Next Class}
  \begin{itemize}
  \item Next Class we will talk about Statistics, how to use them, and
    R.
  \item Please bring your computers, with the ``R'' package installed
    in them (it is free!)
  \item If possible, also bring some text data from your research,
    maybe we will use it!
  \end{itemize}
\end{frame}

\end{document}
