\section{Interaction Effects}

\begin{frame}{}

  \begin{center}
    {\large {\bf
        Second Order Interaction Effects
    }}
  \end{center}
  
\end{frame}

\begin{frame}[t]{What are Interaction Effects?}{}
  When analyzing data from an experiment, a common question of
  interest is {\bf how do the factors affect the response variable?}.\bigskip

  For example, we want to know how do the values of the parameters $F$
  and $CR$ affect the performance of the \emph{Differential Evolution}
  algorithm.\bigskip

  {\bf Main Effects} are how the factors individually affect the
  response variable:\footnote{hypothetical statements}
  \begin{itemize}
  \item Convergence time decreases as we increase the value of $CR$;
  \item Convergence time changes in a parabole with $F$, and is minimal when
    $F \approx 0.5$;
  \end{itemize}
  {\bf Interaction Effects} are how the factors affect the response
  variable in combination;
  \begin{itemize}
  \item When $CR \approx 0.5$, minimal convergence time happens for $F \approx 0.6$
  \item When $CR \approx 0.8$, minimal convergence time happens for $F \approx 1.1$
  \end{itemize}
\end{frame}

\subsection{Simple Example}
\begin{frame}{Simple Example -- Chocolate survey}

  Imagine that we are conducting a survey to discover what is the best
  food condiment.\bigskip

  We perform a survey, where we give the person a random food with a random condiment,
  and ask them their satisfaction.
  \begin{itemize}
  \item Independent Factor: {\bf Food:} Hotdog, Ice Cream
  \item Independent Factor: {\bf Condiment:} Chocolate Sauce, Mustard
  \item Response Variable : Enjoyment
  \end{itemize}

  \includegraphics[width=0.2\textwidth]{../img/irasutoya_hotdog.png}
  \hfill
  \includegraphics[width=0.2\textwidth]{../img/irasutoya_icecream.png}
  \ppagenote{Hotdog and icecream images from irasutoya}
\end{frame}

\begin{frame}[fragile]{Chocolate survey data}

  Here is what the data looks like (you can download data and code from manaba)\medskip

\begin{verbatim}
% cat Condiments.csv
"Enjoyment","Food","Condiment"
81.9269574232529,"Hot Dog","Mustard"
84.9397743293292,"Hot Dog","Mustard"
90.28647932801438,"Hot Dog","Mustard"
89.56180151665502,"Hot Dog","Mustard"
97.67682591880066,"Hot Dog","Mustard"
83.61712996934618,"Hot Dog","Mustard"
89.21086046510206,"Hot Dog","Mustard"
90.7668667221883,"Hot Dog","Mustard"
102.62044030772365,"Hot Dog","Mustard"
85.74390036422551,"Hot Dog","Mustard"
96.5923588947792,"Hot Dog","Mustard"
91.99449143926803,"Hot Dog","Mustard"
87.3332551483422,"Hot Dog","Mustard"
86.87685892829761,"Hot Dog","Mustard"
87.51938536071019,"Hot Dog","Mustard"
93.49125482138082,"Hot Dog","Mustard"
94.30378893191127,"Hot Dog","Mustard"
95.86999176797764,"Hot Dog","Mustard"
81.24348763821509,"Hot Dog","Mustard"
80.53783465646922,"Hot Dog","Mustard"
\end{verbatim}
\end{frame}

\begin{frame}[fragile]{Visualizing the results with a boxplot}
  At first glance, it seems that the condiment does not make a big
  difference?

  {\smaller
\begin{verbatim}
> satisfaction.df <- read.csv("Condiments.csv")
> boxplot(satisfaction.df$Enjoyment ~ satisfaction.df$Condiment)
\end{verbatim}}
  
  \begin{center}
    \includegraphics[width=0.65\textwidth]{../img/boxplot_condiment.png}
  \end{center}
\end{frame}

\begin{frame}[fragile]{Means across factors}{}
  We suspect that something is wrong here, so we plot the means for
  each combination of food and condiment.\footnote{It is always good
    to be curious about your data and explore it in depth.}\bigskip

  {\smaller
\begin{verbatim}
> aggregate(Enjoyment ~ Food + Condiment, satisfaction.df, mean)
       Food       Condiment Enjoyment
1   Hot Dog Chocolate Sauce  65.31661
2 Ice Cream Chocolate Sauce  93.04810
3   Hot Dog         Mustard  89.60569
4 Ice Cream         Mustard  61.30891
\end{verbatim}} \bigskip

    It appears that the Enjoyment depends on both the food and the
    condiment! Unfortunately, ice cream with mustard does not seem to
    be a very popular choice...
\end{frame}

\begin{frame}[fragile]{Measuring Interaction Effects}{}

  We can get a statistical measure of our interaction effect by using
  the ANOVA analysis on a linear model that includes the effect.

  {\smaller
\begin{verbatim}
> Food.lm <- lm(formula = Enjoyment ~ Food + Condiment + Food*Condiment, 
+               data = satisfaction.df)
> # ANOVA (effects of each factor on the final result)
> summary(aov(Food.lm))
               Df Sum Sq Mean Sq F value  Pr(>F)    
Food            1      2       2   0.064 0.80136    
Condiment       1    278     278  11.071 0.00135 ** 
Food:Condiment  1  15696   15696 626.153 < 2e-16 ***
Residuals      76   1905      25                    
---
Signif. codes:  0 ‘***’ 0.001 ‘**’ 0.01 ‘*’ 0.05 ‘.’ 0.1 ‘ ’ 1
\end{verbatim}}
    The F (and P) values on the Food:Condiment row indicate the strong interaction effect.

\end{frame}

\begin{frame}[fragile]{Visualization of Interaction Effects}{}

{\smaller
\begin{verbatim}
interaction.plot(satisfaction.df$Food, satisfaction.df$Condiment, 
                 satisfaction.df$Enjoyment, 
                 xlab = "Food", ylab = "Enjoyment", trace.label = "Condiment")
\end{verbatim}}

\begin{center}
  \includegraphics[width=0.7\textwidth]{../img/interactionplot_condiment.png}
\end{center}
\end{frame}

\subsection{Lung Cancer Example}
\begin{frame}{Another Example: Smoking, Abestos and Cancer}

  The data below was presented in Hilt et al. (1986), about the risk
  of lung cancer depending on whether a person smokes, and whether
  they are exposed to abestos.

  \begin{center}
    \includegraphics[width=0.55\textwidth]{../img/interaction_lungcancer.png}
  \end{center}\ppagenote{Lung Cancer table from VanderWeele and Know: A Tutorial on Interaction}

  We can observe that the risk of Lung Cancer increases much more when
  both Abestos and Smoking are present. More than we would expect from
  either factor alone.\bigskip

  Interaction effects can be both negative and positive.
\end{frame}

% TODO: Improve understanding of additive and multiplicative interaction effects

%\begin{frame}{Measuring Interaction}
%  - Additive Effects
%  - Multiplicative Effects (Ratio Effects)
%  
%  Which to use? Difficult question. There is always ``some'' interaction, if both
%  variables have an effect on the outcome. Better look at both, and consider the
%  characteristics and motivation of the experiment.
%\end{frame}



\subsection{Interaction Effects on DE}

\begin{frame}{Interaction Effects on DE Parameters}{}
  Let's see a more complex example, that is closer to computer science.\bigskip

  {\bf Differential Evolution} (DE) is a popular meta-heuristic algorithm
  that can be used to solve blackbox optimization problems.\bigskip

  The it is known that the performance of DE depends on the problem
  being solved, as well as its two control parameters: $F$ and
  $Cr$. Therefore, the user is encouraged to {\bf fine tune} the
  algorithm prior to application.\bigskip

  In this example, we run a simplified experiment to find optimal
  parameter values for DE for a sample problem.
\end{frame}

\begin{frame}[t]{DE Description}{}
  Differential Evolution is an algorithm that rely on two parameters:
  \begin{itemize}
  \item CR: Ranges from 0 to 1, recommended value is 0.5;
  \item F: Ranges from 0 to 2, recommended value is 0.8;
  \end{itemize}\bigskip

  It is recommended that the algorithm is fine-tuned for specific
  problems. In this case, we will use check the performance of the
  algorithm in a benchmark function under different parameter values.
\end{frame}

\begin{frame}[t]{Experimental Design}{}

  Scientific inquiry: Understand how the convergence rate of DE
  changes as we set different values of the parametes CR and F. In
  specific, try to detect if there are any interactions between these
  parameters.\bigskip
  
  \begin{itemize}
  \item {\bf Response Variable}: Number of iterations until a target solution quality is reached.
  \item {\bf Independent Factor 1}: CR, levels: 0.1, 0.3, 0.5, 0.7, 0.9
  \item {\bf Independent Factor 2}: F, levels: 0.2, 0.4, 0.8, 1.0, 1.2
  \item {\bf Noise Factor}: Random seed (probabilistic algorithm)
  \end{itemize}
  \bigskip

  Since this is an exploratory experiment, we do not set specific
  values for $\alpha$, $\beta$, etc. These values should be set when
  specific null and alternate hypothesis are considered.  
\end{frame}


\begin{frame}[t]{Initial observation of the parameters -- CR}{}
  First, let's observe how changing the parameters affect the response variable:
  \begin{columns}
    \column{0.4\textwidth}
    Notes:
    \begin{itemize}
    \item This is a box plot of the response value, grouped by CR value.
    \item It includes different values of F in the results, so
      interaction effects might be a concern.
    \item Still, we see a generally downwards trend for higher CR values {\bf For this problem}.
    \end{itemize}
    \column{0.6\textwidth}
    \includegraphics[width=\textwidth]{../img/DE_experiment_CR_analysis.png}
  \end{columns}
\end{frame}

\begin{frame}[t]{Initial observation of the parameters -- F}{}
  Same boxplot observation, focusing on the parameter F:
  \begin{columns}
    \column{0.4\textwidth}
    Notes:
    \begin{itemize}
    \item The parameter F shows a clearly non-linear relationship with
      the output variable.\bigskip
      
    \item The anomalous behavior for $F=1$ should be investigated more
      carefully (but we won't do it here).
    \end{itemize}
    \column{0.6\textwidth}
    \includegraphics[width=\textwidth]{../img/DE_experiment_F_analysis.png}
  \end{columns}
\end{frame}

\begin{frame}[t,fragile]{Measuring Interaction Effect with a linear regression}{}

{\smaller
\begin{verbatim}
> DE.lm <- lm(formula = R ~ F + CR + F*CR,            <-- Linear Reg Formula
              data = tab.results)                         Includes F*CR, the
> summary(DE.lm)                                          iteraction factor.
(...)
Residuals:
    Min      1Q  Median      3Q     Max 
-90.107 -36.083   7.101  31.577 103.793               <-- Residuals don't look
(...)                                                     too skewed.
Coefficients:
            Estimate Std. Error t value Pr(>|t|)    
(Intercept)   212.31      12.94  16.409  < 2e-16 ***  
F             -13.15      15.97  -0.823  0.41124      <-- no "linear" effect!
CR           -177.01      22.52  -7.859 1.21e-13 ***
F:CR           74.56      27.81   2.681  0.00783 **   <-- Detects a significant
(...)                                                     effect from changing
                                                          both factors together
\end{verbatim}}                                                 
\end{frame}

\begin{frame}[t,fragile]{Interaction Graph -- CR by F}{}
  A visual investigation of interaction effects might be more informative.
\begin{verbatim}
interaction.plot(tab.results$CR, tab.results$F, tab.results$R)
\end{verbatim}

  \begin{columns}
    \column{0.6\textwidth}
    \includegraphics[width=\textwidth]{../img/interaction_DE_CRbyF.png}
    
    \column{0.4\textwidth}
    \begin{itemize}
    \item $F=0.2$ and $F=1.2$ show low convergence compared to $F=0.4$
      and $F=0.8$. But the shape of the curve seems roughly similar.
    \item Again we see the anomalous behavior of $F=1$
    \end{itemize}
  \end{columns}
\end{frame}

\begin{frame}[t,fragile]{Interaction Graph -- F by CR}{}
  A visual investigation of interaction effects might be more informative.
\begin{verbatim}
interaction.plot(tab.results$F, tab.results$CR, tab.results$R)
\end{verbatim}

  \begin{columns}
    \column{0.6\textwidth}
    \includegraphics[width=\textwidth]{../img/interaction_DE_FbyCR.png}
    
    \column{0.4\textwidth}
    \begin{itemize}
    \item Here we see that the curves show roughly the same shape,
      and convergence speed increasing as CR increases;
    \item Here the anomalous behavior of $F=1$ is more pronounced.
    \end{itemize}
  \end{columns}
\end{frame}

\begin{frame}[t]{Early Conclusions}
  \begin{itemize}
  \item The linear regression model indicates a strong effect for a CR*F factor.\medskip
    
  \item However, this effect is not so clear in a visual inspection.
    \begin{itemize}
    \item This may be because the relationship between F and the
      output variable is not linear.
    \end{itemize}\medskip
    
  \item The recommendation {\bf from these results} would be to try
    $F=0.8$, and $CR$ as high as possible.
    \begin{itemize}
    \item Maybe worth investigation $CR > 0.9$;
    \end{itemize}\medskip
    
  \item Maybe worth it investigating $F = 1.0$ for bugs in the code,
    since a scaling factor should not have such an anomalous behavior.
    \bigskip
    
  \item \structure{A good experiment many times raises as many
      questions as it answers!}
    
  \end{itemize}
\end{frame}

\begin{frame}[t]{Caveats of this case study}{}
  \begin{itemize}
  \item {\bf Non-linear effects}: $F$ shows a clear non-linear
    relationship with the output variable.
    \begin{itemize}
    \item Note how the difference cause the linear model to return
      non-reliable information!
    \item Good opportunity to investigate non-linear models.
    \end{itemize}\bigskip

  \item Second order interaction is already though. Third order
    interaction and above become exponentialy harder to analyze;
    \bigskip

  \item Eventually, it becomes more practical to use sequential models
    like IRACE and SMAC;
  \end{itemize}
\end{frame}


\begin{frame}{Suggested Extra Reading}{}
  \begin{itemize}
  \item Bartz-Bielstein: ``Experimental Methods for the Analysis of
    Optimization Algorithms'', In depth discussion about
    parametrization and analysis of algorithms like DE;\bigskip

  \item Paul Teetor: ``R Cookbook'', Quick recipes for a variety of
    data visualizations in R;\bigskip
    
  \item Statistics by Jim: ``Interaction Effects''.\\
    A short, intuitive description of interaction
    effects. \url{https://statisticsbyjim.com/regression/interaction-effects/}
    \bigskip
    
  \item T.J. VanderWeele, M. J. Know: ``A Tutorial on Interaction.\\
    A more in-depth discussion of interaction in several models, but focused on medicine use cases.
    \url{https://www.hsph.harvard.edu/wp-content/uploads/sites/603/2018/04/InteractionTutorial_EM.pdf}
  \end{itemize}
\end{frame}

%%% Notes
% Interaction effects are common in regression models,
% Anova, and designed experiments
% Interaction plot -- (of two variables)



% Categorical Variables
% Continuous independent variables




%%% Local Variables:
%%% mode: latex
%%% TeX-master: "topic08"
%%% End:
