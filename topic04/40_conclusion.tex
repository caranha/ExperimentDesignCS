\section{Conclusion}
\begin{frame}
  \begin{center}
    {\bf Conclusion}
  \end{center}
\end{frame}

\begin{frame}{Summary}
  \begin{itemize}
    \item In the Last class, we described the Null Hypothesis testing method to perform {\bf Statistical Inference} on a population parameter from a single sample;
    \bigskip

    \item In this cless, we generalize this procedure to a common situation, where we want to compare two samples regarding a population parameter;
    \bigskip

    \item The {\bf Two Sample Hypothesis Testing} uses inference on a statistic based on the \structure{difference between sample estimators}.\bigskip

    \item When there is a high correlation between the observations of each sample, it is important to perform the \structure{pairing} of the observations.
  \end{itemize}
\end{frame}

\subsection{Report 02}

\begin{frame}{Report 2}{Design, Execute and Analize a Scientific Experiment}
  In this report, you must choose a simple experiment to design, perform and analyze the results. Like in Report 1, your report must have:
  \begin{itemize}
    \item {\bf Introduction}: Description of Scientific Question;
    \item {\bf Experiment Design}: Plan of data collection \structure{and analysis}
    \item {\bf Data Collection:} Report on the data and results;
    \item {\bf Analysis:} \structure{Conclusion based on hypotheses and statistic};
  \end{itemize}

  \begin{alertblock}{Important Notes}
    \begin{itemize}
      \item Use techniques of \structure{Statistical Inference} to analyze the results;
      \item In the Experiment Design section, describe the hypotheses used, and the variable of interest;
      \item In the Analysis section, do not forget to test the inferential assumptions (normality, variance, etc);
      \item Remember to follow practices of reproducible science!
    \end{itemize}
  \end{alertblock}
\end{frame}

\begin{frame}{A note about data re-use}
  Because this report is very similar to Report 01, a natural question is "Can I use the same data as in the first report, and just change the analysis?".\bigskip

  \alert{The short answer is {\bf NO}}.\bigskip

  The idea of experimental design is that the setup of the experiment, including the choice of variable, statistic, hypotheses, etc, should guide the data collection, not the other way around. \structure{Choosing the analysis {\bf after} data collection is an easy way to bias the results}.

  \begin{exampleblock}{How you can use your initial data:}
    You can, however, use your previous experiment data to guide the design of the new experiment. For example, you can use it to estimate variance, reasonable values for the hypotheses, etc.
  \end{exampleblock}
\end{frame}

\subsection{Related Reading}
% \begin{frame}{Related Reading}
% \end{frame}
