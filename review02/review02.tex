\section{02 -- Statistical Indicators}

\begin{frame}[t]{Topic 02 -- Statistical Indicators}

  \begin{itemize}
    \item A model is a mathematical description of something that we want to study;
    \item We use information from an experiment to specify aspects of a model;
    \begin{itemize}
      \item The {\bf population} of an experiment is the set of all possible results of that experiment;
      \item An {\bf observation} is a single data point from an experiment;
      \item A {\bf sample} is a set of observations;
    \end{itemize}
    \item A {\bf Statistical Indicator} is a function that uses data obtained from a model to calculate some of its parameters (characteristics).
    \begin{itemize}
      \item For example: using data about the running time of a program, we use the mean as an estimate of the typical running time of that program.
    \end{itemize}
    \item Point Estimators calculate specific values for a parameter, while Interval Estimators calculate a range of likely values.
    \item The value calculated by an Estimator may not be the real value of the parameter;
    \begin{itemize}
      \item {\bf Error} Difference between the value of the estimator and the value of the parameter;
      \item {\bf Bias} Systematic error caused by an Estimator;
    \end{itemize}
  \end{itemize}

\end{frame}

\begin{frame}[t]{Example of Statistical Indicators}
  \begin{itemize}
    \item Mean
    \item Median
    \item Variance
    \item Correlation
    \item Confidence Interval
    \begin{itemize}
      \item Confidence Interval is an Interval Estimator;
      \item It calculates an interval that may contain the true value of the parameter with X probability;
    \end{itemize}

  \end{itemize}\bigskip

  Using Interval Estimators, such as the confidence interval, gives us more
  information about the model than just using point estimators such as the
  mean.
\end{frame}
