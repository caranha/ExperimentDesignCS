\section{01 -- What is an Experiment?}

\begin{frame}[t]{Topic 01 -- What is an experiment?}
  \begin{itemize}
    \item The scientific method is a complex system. It includes not only experimentats and theories, but also community, communication, motivation and interaction with the society;

    \item Experiments are used to obtain data from the world in a methodical fashion;

    \item {\bf Experiment Design} is the discipline of planning how to conduct an experiment to answer a scientific question;
    \item When designing an experiment, we must:
    \begin{itemize}
      \item Choose which data is obtained from the experiment (return variable);
      \item Choose the conditions to execute the experiment (controlled factors and noise factors);
      \item Choose the objectives of the experimental analysis (experimental parameters)
      \item Choose how to obtain the data of the experiment (number of observations, blocks)
      \item Choose how to analyze the data of the experiment (statistical model, visualization)
    \end{itemize}
    \item All these design choices must be defined before the experiment begin.
  \end{itemize}
\end{frame}

\begin{frame}[t]{Topic 01 -- Characteristics of a Good Experiment}
  A good experiment...
  \begin{itemize}
    \item ... examines a falsifiable hypothesis. (the hypothesis has clear criteria of what result would support or reject it)
    \item ... is reproducible. (all information necessary to repeat the experiment is available)
    \item ... controls the experimental environment to minimize the effects of factors unrelated to the scientific question (fairness of comparison; independence of conditions; etc)
  \end{itemize}\bigskip

  {\bf Pre-registration} of experiments can be used to reduce the effect of human bias: The experiment design fully is published before the experiment is executed.

  {\bf Open Data and Reproducible Experiments} are important to allow other scientists to double check and improve your work.
\end{frame}
