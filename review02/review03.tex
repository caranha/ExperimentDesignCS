\section{03 -- Hypothesis Testing}

\begin{frame}[t]{Topic 03 -- Hypothesis Testing}
  \begin{itemize}
    \item {\bf Statistical Inference} procedures use data from an experiment to establish the probability of an statement being true.\vfill
    \item {\bf Null Hypothesis Significance Testing} uses statistical inference to compare different hypothesis about the model under study.
    \begin{itemize}
      \item Null Hypothesis: Standard assumptions about the model. No clear effect.
      \item Alternate Hypothesis: Unexpected effects. Anomalies.
    \end{itemize}\vfill
    \item The Hypothesis test procedure consists of collecting data through experiment, and then using that data to compare the probabilities of the null and alternate hypothesis.\vfill
    \item Possible outcomes:
    \begin{itemize}
      \item Null Hypothesis cannot be rejected;
      \begin{itemize}
        \item Data evidence towards the alternate hypothesis is not strong enough to reject NH;
        \item Type II error: Null Hypothesis is actually false;
      \end{itemize}
      \item Reject the Null hypothesis;
      \begin{itemize}
        \item Alternate hypothesis is more likely than NH by a large margin, given the data;
        \item Type I error: Null Hypothesis is actually true;
      \end{itemize}
    \end{itemize}
  \end{itemize}
\end{frame}

\begin{frame}[t]{Topic 03 -- Hypothesis Parameters}
  \begin{itemize}
    \item $\alpha$: Desired probability of a type I error.
    \begin{itemize}
      \item Test confidence level: 1-$\alpha$
      \item Used as probability threshold for rejecting the null hypothesis;
    \end{itemize}
    \item $\beta$: Desired probability of a type II error
    \begin{itemize}
      \item Test power: 1-$\beta$
      \item Actual probability of a Type II error is controlled also by unknown factors;
    \end{itemize}
    \item $\delta^*$: Minimal Interesting effect size;
    \begin{itemize}
      \item Minimal difference in the experiment result that has practical interest,
      \item {\bf regardless of the hypothesis test result}
    \end{itemize}
    \item $n$: Sample size -- Number of observations \emph{for each experimental condition}
  \end{itemize}
\end{frame}

\begin{frame}[t]{Topic 03 -- Hypothesis Tests}
  \begin{itemize}
    \item {\bf Z test}: Compares the indicator against a fixed value. Calculates the probability of the sample when the Null hypothesis is true. Assumes the sample residuals come from a Normal distribution with known variance; \bigskip

    \item {\bf T test}: Compares the indicator against a fixed value. Calculates the probability of the sample when the Null hypothesis is true. Assumes the sample residuals come from a t distribution with $n-1$ degrees of freedom. Estimates the variance from the sample error.\bigskip

    \item {\bf p-value}: Maximum value of $\alpha$ (lowerst significance level) that would reject the null hypothesis of a test.
  \end{itemize}
\end{frame}
