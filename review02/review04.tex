\section{04a -- Comparison Testing}

\begin{frame}[t]{Topic 04 -- Two Sample Testing}
  \begin{itemize}
    \item {\bf t-test for two samples:} Perform Statistical Inference on the difference between the two samples;\bigskip

    \item Assumptions:
    \begin{itemize}
      \item Normality of residuals;
      \item Equality of variances;
      \item {\bf Independence of Observations}
    \end{itemize}\bigskip

    \item To reduce the effect of different variance, sample sizes proportional to the variance can be used;\bigskip

    \item The independence assumption must be guaranteed at the experiment design stage;
  \end{itemize}

\end{frame}

\section{04b -- Paired Testing}

\begin{frame}[t]{Topic 04 -- Paired Testing}
  \begin{itemize}
  \item The Assumption of Independence states that the experimental conditions of all observations are exactly the same;\bigskip

  \item On the other hand, {\bf Noise Factors} can influence the results of an experiment in a systematic manner;\bigskip

  \item {\bf Pairing} is a technique to reduce the effect of a noise source that affects {\bf pairs of observations} in an experiment equally.\bigskip

  \item The statistical model for paired test is very similar as the model for two sample testing, except on how the difference between samples is calculated;
  \end{itemize}

\end{frame}
